%!TEX root = ../board-prep.parent.tex


\section{循環器}
\subsection{IE}
\begin{itemize}

\item 改訂Duke基準による診断基準について
\end{itemize}


\section{腎臓}

\subsection{AKI(急性腎障害)}
\begin{itemize}
\item AKIは\textbf{血清Cre}と\textbf{尿量}で診断.


\end{itemize}

\section{血液}

\subsection{MDS(骨髄異形成症候群)}
\begin{itemize}

\item 低リスク群MDSの基本は保存加療で,\textbf{RBC輸血}や\textbf{ESA製剤(ダルベポエチン)}投与を行う.
\begin{itemize}
\item 輸血すると症状は大きく改善するが,頻回輸血は\textbf{鉄過剰症}が問題になるので,できるだけ回避したい.
\item ESA製剤は\textbf{血清EPO<500U/mL}でないと効果がない!
\end{itemize}
\item ルスパテルセプト(レブロジル®)の適応は,\textbf{IPSS-Rで低リスク群}\footnote{High, Very Highに対する有効性は確立していない.}かつ\textbf{輸血依存}かつ\textbf{環状鉄芽球}陽性.

\begin{itemize}
\item 「環状鉄芽球陽性」の定義は,①骨髄赤芽球のうち環状鉄芽球\textbf{>15\%} or ②\textbf{SF3B1}遺伝子変異が陽性で,環状鉄芽球\textbf{>5\%}.
\item 環状鉄芽球陰性ではルスパテルセプトの有効性は未確立.\footnote{基本はESAを使うが,EPO>500U/mLだとESAも微妙なので困る.}

\end{itemize}

\insfig{mds_ipssr.jpg}{https://east-hem.clinic/骨髄異形成症候群(mds)}
\end{itemize}

\section{呼吸器}
\subsection{マイコプラズマ肺炎}
\begin{itemize}
\item 典型的には\textbf{寒冷凝集反応}陽性→\textbf{血管内溶血}によるAST↑(ALT→),LDH↑,Bil↑(間接型優位).Na↓.
\item 迅速診断は\textbf{咽頭ぬぐい液}の\textbf{核酸増幅検査(LAMP,\,PCR)}.
\end{itemize}
\subsection{閉塞性肺疾患}
\begin{itemize}
\item COPD患者に\textbf{FeNO↑}(>32ppb),\textbf{Eos増多}があれば\textbf{ACO (喘息/COPDオーバーラップ)}を疑う.
\item 喘息,COPD, ACOいずれも重症例の吸入は\textbf{ICS/LABA/LAMA}が基本.

\end{itemize}

\section{中毒}
\begin{itemize}
\item シアン化水素(HCN)中毒の特徴は,\textbf{火災現場},意識障害,\textbf{顔面紅潮},著明な\textbf{乳酸アシドーシス}.
\item シアン化水素中毒の治療は,\textbf{ヒドロキソコバラミン} or \textbf{亜硝酸塩}.

\end{itemize}

\section{公衆衛生}
\subsection{2024年のヘルシンキ宣言における改訂点}
\begin{itemize}

\item 被験者(human subject)から\textbf{参加者(participants)}に変更された.
\item 研究参加者はボランティアとして(リスクを被りながら)研究に参加する一方で,恩恵を受ける機会は少ないという\textbf{構造的な不平等 (structural inequalities)}に対する配慮が必要である.
\item 参加者の視点や価値判断を知り,それを研究計画に反映させること = 参加者やそのコミュニティの人々の\textbf{「意味のある関与 (meaningful engagement)}.
\item 医学研究の情報,試料リソースの確保と管理 = \textbf{バイオバンク}や\textbf{ビッグデータ}の管理基準.

\end{itemize}


\begin{itemize}

\item アスピリン喘息に伴う好酸球性副鼻腔炎に対しては,経口GCが第一選択.

\item クリプトコッカス髄膜炎に対しては,AMPH-B+フルシトシン2w>フルコナゾール8w>フルコナゾール6-12Mの維持療法(L-AMBのエビデンスレベルはあまり高くない)..


\item 2024年版のNTM暫定基準
\item LTBIの治療:6Hまたは9H (第1選択),3~4HR (第2選択),4R (INHが使用できない場合)
\item 接触者検診でIGRA陽転化していたら治療.
\item IPMNで悪性転化を疑う所見
\item 乾癬性関節炎に対する生物学的製剤:TNFα阻害,IL-12/23,IL-17.PDE4,JAK

\item 不眠症,倦怠感,体重減少には加味帰脾湯

\item 局所麻酔薬アレルギー(エステル型とアミド型は交差反応はしない)

\item 緩徐進行1型DMの治療介入 $https://www.jds.or.jp/uploads/files/article/tonyobyo/66_807.pdf$


\item 無顆粒球章を発症したら抗甲状腺薬の中止,無機ヨウ素への切り替え,ABx治療

\item パーキンソン病の難病申請基準

\item EGPAの臓器障害では末梢神経障害の頻度が最も高い

\item 意識消失発作を伴うVT→PM植込術

\item 侵襲性連鎖球菌感染症の感染症法,インフルエンザに続発?

\item ITPで1万を切ると臓器出血のリスク

\item irAE副腎不全はヒドロコルチゾンから開始

\item 4型RTA→SjS

\item SLEの妊婦→中止すべきはMMF(AZAはOK,ヒドロキシクロロキンはむしろ推奨,t化虚リムスはOK).

\item J-CHS基準によるフレイル診断の項目

\item DPP4iで水疱性類天疱瘡→BP180抗体
\item 逆Gottronは有痛性?MDA5
\item 好酸球増多症による浮腫→四肢のnon-pitting edema IgEは正常のこともある
\item ATLの病型分類(今回はくすぶり型).モガムリズマブの使い方,レナリドミドの使い方
\item 夜の下半身の冷えによる頻尿→牛車腎気丸!
\item MCD(琢ローン性ガンマグロブリン血症,IL-6過剰賛成).HHV-8関連あり?トシリズマブ
\item アドリアマイシン心筋障害は用量依存性の心筋毒性(不可逆的).



\end{itemize}