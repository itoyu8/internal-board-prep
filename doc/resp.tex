%!TEX root = ../board-prep.parent.tex


\section{呼吸器}
\subsection{総論}
\subsubsection{呼吸器の発生と解剖}
\begin{itemize}
\item 前腸(foregut)の腹側に肺芽が出現 → 分岐と伸長を繰り返して肺胞.
\item 胚芽期(26日-6w:気管支まで)→偽腺管期(6-16w:終末細気管支まで)→細管期(16-28w:呼吸細気管支まで,サーファクタント産生開始)→嚢状期(28-36w:間質が減少,サーファクタント分泌完成,胎外生活可能)→肺胞期(36w-).
\item 細気管支(φ2mm)の特徴:気管軟骨と気管支腺が消失,club cell>線毛上皮細胞,Millerの二次小葉を支配.
\item Club cell(旧Clala cell)は分裂能があり,CCSP (club cell secretory protein)を分泌する.
\item 細気管支=二次小葉,終末細気管支=細葉(細葉が集まって二次小葉を形成)
\item 呼吸細気管支の定義:壁に肺胞が付着した細気管支.
\item 終末〜呼吸細気管支から,中枢側に向かって逆行する反回枝(娘枝)が出る.
\item 肺胞の直径は0.1-0.2mm.
\item 気管支動脈は,右は肋間動脈,左は胸部大動脈から分岐.
\item 気管支動脈の血流量は心拍出量の1\%.
\item 胸膜中皮細胞は中胚葉由来,水代謝に関与する.



\end{itemize}


\subsubsection{呼吸生理}
\begin{itemize}

\item 化学受容器は中枢:延髄腹側(PaCO2),末梢:頸動脈小体,大動脈小体など複数.
\item 低酸素換気抑制:低酸素状態が長く続くと換気量↑のレスポンスが鈍くなる.
\item PaCO2↑に対応する換気量upはPaO2↑で鈍る(ただのCO2ナルコーシス),睡眠でも鈍る.
\item 酸素解離曲線の右方シフトは体温↑,アシデミア,2,3-DPG↑.

\item 血管内皮細胞はACEを分泌,ブラジキニンとセロトニンを不活化(分解).
\item アラキドン酸カスケードの起点はcPLA2 (cytosolic phospholipase A2).COX系の脂質メディエーターはPGとTXA2,5-LO系の脂質メディエーターはLT.

\end{itemize}


\subsubsection{疫学}
\begin{itemize}
\item 新規の肺癌:男性8万人,女性4万人.
\item 塗抹陽性結核:4例/10万人.
\item 結核死:2000人/年.90歳以上の結核患者の死亡率:50\%.LTBI患者のうち医療従事者:25\%.
\item 喘息死:2000人/年(減少),COPD死:18000人/年.
\item 小児喘息のうち成人喘息への移行:30\%.


\end{itemize}

\subsubsection{主要徴候と身体所見}
\begin{itemize}
\item Miller \& Jones分類は肉眼所見,P2以上でgood quality.

\item Geckler分類は顕微鏡所見,4/5群(吸引検体なら6群)でgood quality.

\item ACT:\textbf{20-24点}でコントロール不十分,\textbf{<20点}でコントロール不良.MCIDは3点.小児にはC-ACT.
\item ACTの項目:日常生活への支障,息切れ,夜間の中途覚醒,SABA使用回数,自身での喘息コントロールの自覚.
\item 修正Borgスケール:0〜10,0.5 =「非常に弱い息切れ」.
\item mMRCグレード3:「100mまたは数分歩いて息切れ」.

\insdualfig{dyspnea_fhj.jpg}{1}{dyspnea_mmrc.jpg}{1}{倉原先生のブログより}
\item 嗄声をきたす癌:甲状腺癌>肺癌>食道癌.
\item ばち指:DPD/IPD>1.0(爪甲基部の厚みの方がDIP関節の厚みよりも分厚い = これがばち指の特徴).シャムロス徴候+.指末端でPDGFやVEGFが分泌.
\item 肺性肥大性骨関節症はAd(やSq)に合併.ばち指,四肢長管骨の骨膜新生,関節炎.
\item 抗VGKC抗体:SCLC,胸腺腫.抗VGCC抗体:Lambert-Eaton症候群(つまりSCLC).
\item MGで抗MuSK抗体陽性なら胸腺切除術は非推奨(抗AChR抗体陽性ならDo).
\item Hoover徴候:COPD,振子呼吸:肺結核後遺症.

\end{itemize}
\subsubsection{検査・治療}

\begin{itemize}
\item プリックテストの判定は15分後.H1bは検査4-5日前,H2bは24時間前に中止.LTRAは中止不要.
\item 皮内テストはプリックテスト陰性例に行う.0.02mLを前腕屈側に注射.15分後に判定.

\item 喀痰細胞診は常温で12時間以内,冷蔵で24時間以内に検査する.
\item 被曝量:CXp1回で0.04mSv, 胸部CT1回で7.8mSv.

\item 良性石綿胸水の胸水はEosino優位.MPMの胸水はヒアルロン酸>10万ng/mL,SMRP上昇(>8-15nmol/L).


\end{itemize}


\subsubsection{PFT}

\begin{itemize}

\item 肺癌手術の目安:術前\%FEV1>50\%,術前\%DLco>50\%.術後予測1秒量>800mL,術後予測\%1秒量>30\%.
\item DLco測定時の混合ガス:N2, O2, He, CO(COは拡散+希釈,Heは希釈をみる).
\item DLcoの注意点:①smokerは低めに出る(検査前24時間は喫煙),②食後2時間,運動直後を避ける.
\item DLco/VA:ガス交換面積あたりのDLcoを見る → COPD(気腫型)ではDLcoより低めに出る,IPではDLcoより高めに出る.
\item アストグラフ:BAの気道過敏性試験.メサコリン吸入.
\item CV (closing volume):末梢気道閉塞を見る.100\% O2を吸うと肺底部に溜まって肺尖部は空気(N2)が相対的に多い→息を吐くとまず下肺野のO2が先に出る→さらに息を吐いてN2が増え始めたところで下肺野は空気が抜けて(or 末梢気道病変のために)つぶれたと考える.\item CVは普通10\%,20-25\%を超えると末梢気道閉塞.
\item 炭酸ガス換気応答テスト (VR-CO2):呼気を繰り返し吸わせてCO2↑→普通は換気量up.先天性中枢性低換気症候群 (CCHS) では換気量upの反応がなくなる.

\end{itemize}

\subsubsection{BFS}

\begin{itemize}
\item BALの正常所見:細胞数〜13万,Mph 90\%,Ly 10-15\%,CD4/8 1-2.
\item BALのCD4/8比↑:サルコイドーシス,結核,ベリリウム肺,慢性HP.CD4/8比↓:CTD-ILD(NSIP, COP),急性〜亜急性HP,珪肺??


\end{itemize}

\subsection{疾患}

\subsubsection{SAS}

\begin{itemize}
\item エプワース眠気尺度:>11点で高リスク → 簡易PSG(アプノモニター)に進む.

\insfig{epworth_scale}{}{https://hirai-naika-geka.com}

\item 簡易PSGでは純睡眠時間は分からないので,代わりにモニター装着時間を使う → AHIではなくREI (respiratory event index).普通にAHIより低く出る.
\item PSGの記録チャンネル7個:脳波,眼電図,筋電図,心電図,気流,呼吸努力,SpO2.

\item OSA重症度(AHI):軽症5-15,中等症15-30,重症>30.
\item CPAP保険適応基準:アプノモニターでAHI>40,PSGでAHI>20.

\end{itemize}

\subsubsection{細菌感染症}

\begin{itemize}

\item A-DROPはSpO2<90\% (r/a),I-ROADはSpO2<90\% (FiO2 35\%).
\item CURB65:年齢>65,呼吸数>30(SpO2ではない).残りはA-DROPと同じ.
\item NHCAP:①施設,②<90日以内に退院,③要介護,④定期的な血管内治療(透析,ケモ,免疫抑制薬,ABx).
\item NHCAPもA-DROP使う.

\insdualfig{adrop.jpg}{1}{iroad.jpg}{0.9}{https://knowledge.nurse-senka.jp}

\item 多剤耐性緑膿菌(MDRP)の定義:カルバペネム,アミノグリコシド,キノロンに耐性.
\item 耐性グラム陰性桿菌(CREなど)の治療薬:コリスチン,チゲサイクリン,セフィデロコル(フェトロージャ\circledR),セフトロザン・タゾバクタム(ザバクサ\circledR),イミペネム・シラスタチン・レレバクタム(レカルブリオ\circledR)
など.
\end{itemize}

\subsubsection{結核}

\begin{itemize}
\item T-SPOTの偽陽性はM.kansasii(有名), M.szulgai, M.marinum, M.gordonae.
\item RFPとINHは減感作療法.
\item RFP+VRCZ併用禁忌.INHはヒスチジンが蓄積して発疹出やすい.

\end{itemize}

\begin{itemize}
\item MAC症に対する標準的なレジメンの確認
\item マイクロ波凝固療法は純酸素でも使用可.APCは気道穿孔リスク低い.
\item BTが作用するのは平滑筋細胞(減らす),線維芽細胞(リモデリング改善),迷走神経.
\item LTBIの治療


\end{itemize}

\subsubsection{BA}

\begin{itemize}

\item Bio適応疾患:慢性蕁麻疹はゾレア+デュピクセント(結節性痒疹はデュピクセントonly),慢性副鼻腔炎はヌーカラ+デュピクセント,EGPAはヌーカラ+ファセンラ.
\item ファセンラはEGPAに対しては在宅自己注射OK!!!

\insfig{ba_bio.jpg}{0.7}{https://medical.nikkeibp.co.jp/leaf/mem/pub/report/202506/589032.html}

\end{itemize}

\subsubsection{肺癌}
\begin{itemize}
\item OncomineだけOK:HER2(エンハーツ,ゾンゲルチニブ)
\item Oncomineだけ不可(Amoy/肺がんCPはOK):ルマケラス
\item AmoyだけOK:新規ROS1(タレトレクチニブ,レポトレクチニブ),新規MET(グマロンチニブ)\footnote{コンパニオン診断は最新情報を確認 https://hokuto.app/post/ZXTQ93WlRz5xYEUeiWoG}
\item 肺がんCPだけ不可(Amoy/OncomineはOK):Ami/Laz,ローブレナ,ロズリートレク,タブレクタ
\item NTRKのロズリートレク,ラロトレクチニブはF1のみコンパニオン診断OK.


\item G-CSFの一次予防はFN発症率>20\% 

\item 2型肺胞上皮細胞と血管内皮細胞は放射線感受性が高い
\item 百日咳の診断方法
\item 院内肺炎の定義 48時間以降.NHCAPの定義
\item A-DROPの
\item NHCAPまでA-DROPで判定.院内肺炎はI-ROAD.耐性菌リスク因子は過去90日以内のiv ABx,過去90日以内の2日以上の入院歴,免疫抑制薬,活動性低下(PS3以上,BArthel 50未満,歩行不能,経管栄養またはCV)
\item 高齢者における肺炎球菌ワクチンの接種スケジュール
\item 誤嚥性肺炎における嚥下機能評価:簡易嚥下誘発試験のやり方について
\end{itemize}

\begin{itemize}

\item リポイド肺炎,死亡貪食マクロファージ,外因性肺炎,ガソリンや灯油,好物など.4割無症状.TBBで判定.対症療法.
\item マイコプラズマは抗原,LAMP,抗体はPA法,ただしペア血清が必要.培養するならPPLO.IgM抗体は発症7日以降.MLs効かなければTCまたはNQ
\item クラミジア肺炎は潜伏期間3-4w,5類感染症(定点把握),IgM(>10日)/IgGのペア血清で診断.
\item レジオネラはGiemenez染色・アクリジンオレンジ染色.培養はBCYE-αやWYO培地

\item 細菌性肺炎と非定型肺炎の違い
\item CPAの血清診断基準.IPAのCTでは病初期ではhalo sign,回復期ではair crescent signが特徴的である.
\item 肺クリプトコッカス症は届出不要,播種性クリプトコッカス症は5類感染症.Cryptococcus neoformansがほとんどだがC.gattiの報告例もある.治療は基本的にフルコナゾールで播種病変がある場合はAMPH-B+5-FC(レジメンもう一度確認,L-AMBのエビデンスは少ない)
\item ムーコル症の場合はAMPH-B, L-AMB, PSCZなどを用いて,可能であれば外科治療も行う.
\item アゾール系の中で血中濃度測定が必要なものとそうでないもの.

\item Tb: 培養陽性の場合核酸同定検査(DDH)などで判定する
\item Tbでの治療延長を検討する条件(9ヶ月):再治療例,重症例,初期2ヶ月の治療後も培養陽性の場合.
\item INH, RFPに耐性の場合は多剤耐性結核.NQと,(カナマイシン,AML,カプレオマイシンのいずれかに耐性)→超多剤耐性結核.
\item 多剤耐性結核に対するレジメン
\item NTMの臨床診断基準
\item M.absesscusの中でmassilienseはerm41(MLs誘導耐性)がないのでMLs使える.

\item 肺吸虫症は日本では宮崎肺吸虫症とWesterman肺吸虫症がツートップ.気胸,浸潤影,結節,空洞,胸水など.1st choiceはプラジカンデル.
\item トキソカラ症は牛,鶏の生食で感染.アルベンダゾール or メベンダゾール.
\item 糞線虫症は沖縄,奄美に多い.ARDS合併,イベルメクチン.

\item PCPの栄養体は、Wright-Giemsa染色法や、その簡易法であるDiff-Quik法で染色することが可能ですが、嚢胞体は染色できません。メテナミン銀染色やトルイジンブルーO染色は嚢胞体を見つけるのに使用されます。

\item Actinomyces, NocardiaはともにGPR.NocariaはZ-N染色で染色される.Actinomycesは胸壁に病変が進展して瘻孔形成しうる.Nocardiaは脳膿瘍合併.


\item 中皮腫はヒアルロニダーゼ消化試験陽性,微絨毛がみられる,CEA陰性

\item COPDの日本の死亡者数は18000人くらい,世界だと死亡第3位まで上がる.喫煙者のうち15-20\%が罹患,疾患関連遺伝子としてSERPINE.IL-17A,IL-1β,IL-6,TNF-αなどが関連.MMP↑,プロテアーゼ活性↑,好中球エラスターゼ↑.Tc1型CD8+T↑,Th1型CD4+↑,Th17型CD4+↑.
\item BA合併は15\%.
\item COPDの病期分類.
\item スパイロのV50/V25の定義とみかた.
\item 喘息の気管支拡張薬使用後の診断基準.
\item 広域集波オシレーション法について.呼吸抵抗(レジスタンス)がどうなるか?47と48
\item COPD assessment test CAT
\item テオフィリンの血中濃度は5-15μg/mL,消化性潰瘍
\item COPDに対するNPPV導入基準
\item COPDの肺容積減量手術は上葉への気腫偏在
\item COPDの空気のCT値は-1000HU
\item COPDの予防として推奨されるワクチン
\item ACO診断基準(日本呼吸器学会)
\item AATの遺伝形式,治療(AATインヒビター点滴).
\item BEの背景はRA, SjD, GERD, PCDなど.PCDはAR,PCDスクリーニングはサッカリンテスト(人工甘味料を使い味を感じるまでの時間の遅延),鼻腔内NO異常低音.
\item BEのCTでは気管支内腔が隣接PAの1.5倍以上に拡大.Pseudomonas検出は予後不良.\%FEV1<50\%も予後不良.
\item BEの重症度評価としてFACED score, BSI.
\item BOは肺野透過性亢進,中枢気道の気管支拡張は進行期までおこらない.air trappingの検出のため呼気CT.NO2, SO2, アスベスト吸入と関連,アマメシバ,薬剤性としてDペニシラミン,RS, アデノ,マイコなどもある.SjS,悪性リンパ腫.constructive bronchiolitis, 病変間に正常気道が介在.根治治療は肺移植のみ.造血幹細胞移植後の慢性GVHDの一病型として現れる(5-10\%,末梢血移植の方が多い),急性GVHDあるとリスク高い,なので造血幹細胞移植後は定期的なPFTが必要(中央値14m),肺移植を行ってもBO起こる(50-60\%).
\item 臓器移植後に閉塞性細気管支症候群(BOS)起こる,1秒量低下をチェックする,BALで細胞数増多と好中球増多,ステロイドも基本は効果なし.

\item DPB, HLA-B54, 典型例は慢性副鼻腔炎合併,好中球性の炎症.MUC5AC(ムチンのコア蛋白)の増加.IgA上昇?寒冷凝集素価上昇(64倍以上),増悪は肺炎球菌やHibが多い.


\item 肺移植適応疾患.
\item 肺嚢胞と肺気腫は別.喫煙歴との関連はない.SjSと肺嚢胞.スキューバダイビングは制限.

\item BAではLTC4, TGF-βによる気道平滑筋肥厚.PGE2は気道に対して拡張的に働く.
\item Th2細胞から賛成されるIL-4,5,13, 好酸球から分泌される???好酸球から分泌されるMBPは気道過敏性を更新させる.
\item IL-33, TSLP→ILC2↑→IL-5↑IL-13↑
\item ILC2はステロイド抵抗性.
\item 粘液細胞からMUC5B, 杯細胞からMUC5ACが分泌されて気道上皮の表面の上層(ゲル層)を構成する.MUC5ACが喘息への病態を悪化させる.
\item ダニとハウスダストのIgEは95\%以上かぶる.舌下免疫療法の適応は5歳以上.喘息の増悪因子の最大の原因はダニ.
\item FeNOの増悪リスクとの関連性はない.ピークフローとは努力呼出時の最大呼気流量.PEF日内変動が20\%以上あれば喘息診断.
\item 気道可逆性の評価基準.EV1が吸入前と比較して20\%以上低下すると要請.

\item AERDは1:2で女性に多い,好発は20-40歳.IgEを介さない機序.周術期はFEV1を予測値あるいは自己最良値の80\%以上まで改善させておく.

\item 好酸球性肺炎のメディエーターとしてeotaxin.

\item IPFのBAL所見,本邦における重症度分類,難病指定のための要件,GAPモデル
\item PFDの副作用は光線過敏,NTDの副作用は下痢,消せ戦塞栓症

\item AE-IPFの診断基準
\item 肺炎球菌ワクチンのスケジュール

\item IPAFの診断基準.臨床ドメインと血清学ドメインと形態学ドメイン Raynaudなどは臨床ドメインか?
\item COPは性差なし.non-smoker多い.CD4/8低下.
\item PPFEはCYなどの悪性腫瘍,GVHDとして発症

\item ARDSのベルリン定義:侵襲または呼吸器症状出現から1習慣以内,両側性の陰影,PEEP 5cm以上でP/F<300,200と100をカットオフで中等症と重症.小児では片側でもARDSで診断.
\item ARDSは予測体重を用いて6-8mL/kgの換気.筋弛緩は<48hrで終了.好中球エラスターゼ阻害薬は一応保険収載.


\item RAではBOありえる.SjDではMALTomaありえる.SLEで肺胞出血の場合はPE.
\item IgG4RD診断基準.
\item サルコイドーシスでCD4/8↑.
\item LCHでCD1a陽性細胞,S-100蛋白陽性.
\item GPAでステロイド+CY.
\item アミロイドーシスの胸膜病変で胸水貯留
\item maltomaの合併:H.pylori, SjD, 橋本病

\item 生着症候群は自家移植.30-100日だとCMV, トキソプラズマ,ARDS,リンパ増殖性疾患,PH.100日すぎるとBOやPPFE

\item 超合金肺の原因はコバルト,タングステン
\item 石綿肺→10-20年経って生じる.良性石綿胸水()→びまん性胸膜肥厚.いずれも高濃度暴露.
\item 低濃度暴露で生じるのは胸膜プラーク→MPM

\end{itemize}

\subsubsection{石綿とMPM}
\begin{itemize}
\item 石綿の高濃度曝露→良性石綿胸水,びまん性胸膜肥厚.低濃度曝露→胸膜プラーク,MPM.
\item 良性石綿胸水の定義:数ヶ月で自然軽快,胸水確認後3年は悪性腫瘍がない(両側に生じてもOK).

\end{itemize}


\subsubsection{じん肺の管理}
\begin{itemize}
\item じん肺の管理は管理1-4に分類され,管理2以上のものは健康管理手帳,公費で健診受けられる.
\item じん肺の診断時にDrが行う検査は①胸部Xp ②PFT → 労働局へ提出.
\item \textbf{じん肺一般の}労災基準:管理4全員,管理2/3でTb/Tb胸膜炎,続発性気胸/気管支炎,気管支拡張症, 肺癌.

\item \textbf{石綿の}労災基準:じん肺の労災基準を満たす場合(じん肺ではなく特に石綿肺と呼ぶ),中皮腫,良性石綿胸水,びまん性胸膜肥厚,肺癌(石綿小体の数が重要なことがありBALはできるだけ行う)

\item 石綿関連疾患で職業曝露以外だと労災は受けられないが,石綿健康被害救済制度の対象となる.MPM,石綿肺癌,良性石綿胸水,びまん性胸膜肥厚いずれも適応になる.

\item 立位では肺尖部で肺胞気圧>肺AV圧.TXA2は血管収縮作用.基本的に徐脈の治療→β刺激薬→気管支拡張作用
\item 有効血管床の50-60\%が障害されるとPA圧上昇
\item 肺性心のECG所見:V1-V3のR波増高,V4-V6深いS波,II, III, aVFのP波高.
\item PE: S1Q3T3
\item 高地肺水腫→低酸素性の肺血管攣縮.神経原性肺水腫→交感神経活性化による毛細血管圧↑
\item PAH: sGC↓エンドセリン↑eNOS(NOが血管拡張作用)↓PDE-5↑PGI2↓
\item 骨髄増殖性疾患によるPHは5群.薬剤性PHの原因:食欲抑制薬(網のレックス,フェンフルラミン),ダサチニブ.PAHであればまずHIV,AUS(門脈圧亢進)
\item PHがあれば常にHOTOK???
\item PAH治療薬のまとめ
\item 3群PHには吸入トレプロスチニル,SSc-PHにも血管拡張薬,他は原疾患の治療から
\item Well's criteria 急性PEには初手はt-PAはあり,カテーテルはなし
\item cancer VTEに対するDOACsの適応について
\item CTEPH 適切な抗凝固治療を3-6m継続しても慢性的にPHが継続している場合にCTEPHと診断.70歳台発症ピーク.DVTや血栓素因などで肺塞栓を反復的に繰り返す.永続的な抗凝固治療が必要.外科的治療は内膜摘除 or バルーン拡張.cTEPHにはワルファリンとリオシグアトは使える!!(最新は?)
\item 遺伝性出血性毛細血管拡張症:粘膜や皮膚の毛細血管拡張病変が特徴,肺動静脈奇形がある場合はダイビングNG.門脈圧亢進症や肝AV瘻によるPHもありえる
\item 肺葉内肺分画症(正常肺葉内に存在,PVに還流,気道感染反復),肺葉外肺分画症(固有の臓側胸膜を持つ,合併奇形あり,無症状),

\item 組織学的効果判定のgrade
\item 免疫染色態度,CD7+CK20-TTF1+,CK7-CK20+だと大腸がんを疑う.p40+, CL5/6+だと扁平上皮癌疑い
\item 炎症性偽腫瘍でALK陽性
\item CT followの頻度について
\item OSASはエプワース眠気尺度,ベルリン質問し,STOP質問しなど
\item PSGはAHI>20,簡易モニターでAHI>40でもOK
\item CPAPでも日中の眠気が続く場合はモダフィニル使用できる
\item PSGで何をモニターしているか:脳は,心電図,筋電図
\item CPAPは4時間以上を目標
\item CSAだと酸素,CPAP,CPAP装着下でAHI>15ならASV(adaptive servo ventilation)を考慮する
\item LAM細胞のクラスターは乳び胸水や乳糜腹水にみられる.LAM細胞はD2-40,α-SMA,HMB45,ER,PRなどの免疫染色が有用.血清VEGF-D
\item BHDS 顔面,頸部,上半身に繊維毛包腫.肺嚢胞は下肺野縦隔側,腎腫瘍はchromophobe腫瘍またはoncocytoma.folliculin遺伝子の異常でAD.
\item 先天性PAPの原因としてSP surfactant protein-B, SP-C, ABCA3遺伝子異常.PAPはBAL液中に泡沫状マクロファージ,末梢気腔内にPAS染色,SP-A染色陽性の再k粒状物質がみられる.PAPはGM-CSF吸入療法(保険適用)

\end{itemize}




\subsubsection{稀な疾患群}

\begin{itemize}

\item Platypnea-orthodeoxia症候群:心内シャント(卵円孔開存),V-Q mismatch(肝肺症候群,IP).

\item 線毛不動症候群の検査はサッカリンテスト陽性(下鼻甲介に置く,咽頭で甘みを感じるまでに時間かかる),鼻腔NO低値.
\item 黄色爪症候群(Yellow nail syndrome, YNS):黄色爪,四肢(下腿)リンパ浮腫.呼吸器症状の合併が多い(BE, 慢性気管支炎).ブシラミン,金,D-ペニシラミンと関連.
\item YNSの胸水:リンパ球優位,TG上昇.
\item YNSの治療は対症療法のみ:MLs, vitamin B3/E, 亜鉛.
\item 

\item MCDはVEGF↑.AAアミロイドーシス合併,HHV-8
\item 肝肺症候群:肺内シャントの存在はコントラスト心エコーが有用,3心拍以内にLAに気泡が到達した場合は心内シャントの存在が示唆される.
\item 肺胞出血の薬剤:抗癌剤(MTX,マイトマイシンC,CyA),抗てんかん薬(フェニトイン,カルバマゼピン),抗不整脈薬(アミオダロン,キニジン)

\item GPA palisading granuloma 中心部の壊死に対して柵状に組織球と巨細胞が取り囲む所見
\item 肺胞微石症:SLC34A2(リンの運搬蛋白),AR,リン酸カルシウムを主成分とする微小結石が蓄積
\item CF CFTR遺伝子変異,汗の塩化物イオン濃度上昇,膵臓の外分泌機能不全,慢性副鼻腔炎,LC,先天性両側精管欠損症など合併
\item 喫煙は自然気胸の最大のリスク
\item 胸水ADA上昇 Tb, 農協,RA, リンパ腫,IgG4RD
\item Tb胸膜炎は一次結核に多い
\item 胸管損傷による乳び胸は胸管の破綻が第5胸椎より上で生じると左胸水,それより下だと右胸水になる.オクトレオチド.排液量が1000mL超える場合は胸管塞栓術やっていい.乳び胸の胸水はリンパ球優位.乳糜状況水(chyliform effusion)は胸腔内での炎症の遷延で生じる,CMは増加しない

\item MPMは胸膜80-85\%,腹膜10-15\%,MPMの70\%で石綿暴露あり
\item MPMのセルブロックの免疫染色.carretinin, WT-1, D2-40, CD5/6.MPMとと反応中皮過形成の鑑別は間質,脂肪組織への浸潤の有無を評価
\item desminによる免疫染色で,非腫瘍性中皮で80\%が陽性,上皮型中皮腫で10\%が陽性.
\item MPMの手術:壁側胸膜と肺表面の臓側胸膜のみを切除

\item 肺吸虫症はイノシシ,ウエステルマンが最多,腹腔→胸腔→気胸・肺野病変
\item 膿胸関連リンパ腫(PAL):慢性膿胸,結核性胸膜炎への人工気胸術後>20年,胸腔内に発生するDLBCL.
\item 原発性滲出性リンパ腫(PEL):HIV関連,HHV-8.
\item MPMガイドライン読む


\end{itemize}
\subsubsection{胸腺腫,胸腺癌}
\begin{itemize}
\item 胸腺腫+低γグロブリン→Good症候群で予後不良.
\item 胸腺腫の正岡分類:IIB期で肉眼的浸潤,IVA期で胸膜・心膜播種,IVB期でリンパ行性,血行性転移.
\insdualfig{thymoma_masaoka.jpg}{1}{thymoma_who.jpg}{1}{朝倉内科学(第12版)}\item 切除可能な胸腺腫・胸腺癌に針生検は禁忌.
\item 胸腺腫はIII期まで1st choiceは外科的切除.不完全切除例はPORTを行う(ただしIII期胸腺腫は完全切除でもPORTやってもいい).


\item 胸腺癌もIII期まで1st choiceは外科的切除.II-III期は完全切除でもPORT必須.
\item 切除不能胸腺腫は化学療法(ドキソルビシン,シスプラチン,CY±ビンクリスチン).ICIは治療関連死の可能性あり不可!
\item 切除不能胸腺癌は化学療法(CBDCA+PTX or CBDCA+AMR).2nd-lineはレンバチニブ.








\item セミノーマはhCG上昇(10-20\%),非セミノーマはAFP上昇(50-70\%).

\end{itemize}
\subsubsection{胸郭疾患}
\begin{itemize}
\item 横隔神経麻痺:第3-5頚椎から出る.診断は臥位Xp.
\item 横隔膜弛緩症:先天性は胎生期の形成不全.後天性は横隔膜の鈍的障害.左の挙上が多い.
\item 漏斗胸の手術適応の判断はHaller index:(胸腔横径)÷(胸骨後面と椎骨前面の距離)>3.25.


\item 大量血胸の開胸止血の適応:①胸腔ドレナージ開始時に1000mL,②開始1時間で1500mL,③開始2-4時間で>200mL/h,④持続的な輸血が必要.ただし鈍的外傷は開胸不要なことも多い.

\end{itemize}


\subsubsection{禁煙外来}

\begin{itemize}
\item 禁煙治療の条件:①禁煙意思あり,②ニコチン依存症スクリーニング>5点,③BI>200(35歳未満は制限なし),④文書による同意(+過去1年以内の禁煙治療歴なし)
\item バレニクリン(チャンピックス\circledR)は漸増して12週使用.最初の1週間だけ喫煙OK!ドライバーには処方不可.
\item ニコチンパッチ(ニコチネル\circledR)は漸減して8週使用.
\item 日本の禁煙治療の成功率は20\%.
\end{itemize}
