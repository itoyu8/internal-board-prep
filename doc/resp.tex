%!TEX root = ../board-prep.parent.tex


\section{呼吸器}
\subsection{病態生理}
\subsubsection{呼吸器の発生と解剖}
\begin{itemize}
\item 前腸(foregut)の腹側に管状の\textbf{肺芽}が出現し,分岐と伸長を繰り返して肺胞が形成される.
\item 胚芽期(26日-6w:気管支まで)→偽腺管期(6-16w:終末細気管支まで)→細管期(16-28w:呼吸細気管支まで,サーファクタント産生開始)→嚢状期(28-36w:間質が減少,サーファクタント分泌完成,胎外生活可能)→肺胞期(36w).
\item 細気管支(φ2mm)では気管軟骨と気管支腺が消失する.壁に肺胞が付着した細気管支を\textbf{呼吸細気管支}とよぶ.
\item 細気管支の末梢側では\textbf{線毛上皮細胞}よりも\textbf{club cell(Clala cell)}の割合が増える.\textbf{CCSP (club cell secretory protein) }を分泌し,分裂能を有する.
\item 終末細気管支〜呼吸細気管支からの分岐において,気管の中枢側に向かって逆行する\textbf{反回枝(娘枝)}が存在する.
\item 肺胞の直径は\textbf{0.1-0.2mm}.
\item Millerの二次小葉の支配気管支は\textbf{細気管支}.細葉は\textbf{終末細気管支}以下に付属するひとまとまりで,二次小葉は複数の細葉から構成される.
\item 気管支動脈は,右は\textbf{肋間動脈},左は\textbf{胸部Ao}から分岐する.気管支動脈の血流量はCOの\textbf{1\%}程度.
\item 胸膜中皮細胞は\textbf{中胚葉}由来で,\textbf{水}代謝に関与する.



\end{itemize}


\subsubsection{呼吸生理}
\begin{itemize}

\item 中枢化学受容器は\textbf{延髄腹側}に存在し,\textbf{PaCO2}のセンサー.
\item 末梢化学受容体は\textbf{頸動脈小体}や\textbf{大動脈小体}など複数存在する.
\item PaO2<60Torrで換気が増強するが,高度の低酸素状態が続くと換気量が減少する(\textbf{低酸素換気抑制}).
\item PaCO2に対応する換気↑はPaO2が低いほど強く応答する.睡眠時は応答が減弱する.
\item 酸素解離曲線の右方シフトは\textbf{体温↑},\textbf{アシデミア},\textbf{2,3-DPG↑}.

\item 血管内皮細胞は\textbf{アンジオテンシン変換酵素}を分泌し,\textbf{ブラジキニン}を不活化,\textbf{セロトニン}を分解する.
\item アラキドン酸カスケードの起点となる酵素は\textbf{cPLA2(cytosolic phospholipase A2)}.脂質メディエーターのうちCOX系の代謝物はPGとTXA2で,5-LO系の代謝物はLT.

\end{itemize}


\subsubsection{疫学}
\begin{itemize}
\item 新規診断される肺癌患者は男性8万,女性4万.
\item 喀痰塗抹陽性の肺結核の罹患率は4例/10万.
\item 結核死は年間2000人程度で横ばい.90歳以上の結核患者の死亡率は約50\%.喘息死も2000人程度だが減少.
\item LTBI患者のうち医療従事者は\textbf{25\%}.
\item 小児喘息のうち成人喘息への移行は\textbf{30\%}.
\item COPD死は年間18000人.


\end{itemize}

\subsubsection{主要徴候と身体所見}
\begin{itemize}
\item Miller \& Jones分類で検査に適した検体は\textbf{P2}以上.
\item Geckler分類で検査に適した検体は\textbf{4群}または\text{5群}で,いずれも好中球数は\textbf{25個以上},上皮細胞は\textbf{25個未満}.
\item ACTは\textbf{20-24点}で不十分,\textbf{<20点}で不良.
\begin{itemize}
\item 項目は①日常生活への支障,②息切れ,③夜間の中途覚醒,④SABA使用回数,⑤自身での喘息コントロールの自覚.
\end{itemize}
\item 修正Borgスケールは\textbf{0から10}までで呼吸困難の程度を表し,「非常に弱い息切れ」は\textbf{0.5点}.
\item mMRCではグレード1は\textbf{平坦な道または緩やかな上り坂}で息切れ,グレード2は\textbf{平坦な道でも息切れ},グレード3は\textbf{100mまたは数分歩いて息切れ}.
\item Platypnea-orthodeoxia症候群の原因として\textbf{卵円孔開存}などの心内シャントや,\textbf{肝肺症候群}や\textbf{間質性肺炎}によるV-Q mismatchなどがある.
\item 嗄声の原因となる悪性腫瘍の頻度は甲状腺癌>肺癌>食道癌.
\item ばち指では\textbf{DPD/IPD>1.0}(爪甲基部の厚みの方がDIP関節の厚みよりも分厚い = これがばち指の特徴).シャムロス徴候陽性.指末端においてPDGFやVEGFが分泌される.
\item 肺性肥大性骨関節症はばち指,四肢長管骨の骨膜申請,関節炎.AdやSqに合併.
\item 抗VGKC抗体は\textbf{SCLC}や\textbf{胸腺腫}と関連.抗VGCC抗体は\textbf{Lambert-Eaton症候群}の原因となり,\textbf{SCLC}と関連.
\item MGの中で抗MuSK抗体陽性の場合,胸腺切除術は推奨されない.
\item Hoover徴候はCOPD,振子呼吸は肺結核後遺症でみられる.

\end{itemize}
\subsection{検査}

\begin{itemize}
\item プリックテストの判定は15分後
\item 皮内テストは0.02mLを前腕屈側に注射
\item T-SPOTの偽陽性はM.kansasii, M.szulgai, M.marinum, M.gordonae
\item 喀痰細胞診は常温で12時間以内,冷蔵で24時間以内
\item レントゲン1回の被爆は0.04mSv, 胸部CT1回の被爆は7.8mSv前後.
\item 黄色爪症候群の胸水リンパ球分画

\item 呼吸機能検査におけるclosing volumeの意義
\item アストグラフは気道抵抗の測定
\item 高二酸化炭素応答テストは呼気CO2と換気量を測定
\item Good症候群は胸腺腫,Kartagener症候群はサッカリンテスト??
\item SASのモニターは7チャンネル.AHIと重症度判定,CPAPの開始基準について
\item バレニクリンは漸増して12w使用する.使用中は運転できない
\item 本邦は禁煙補助薬を使用した禁煙治療の率は低い 20\%くらい
\item AERD患者にはリン酸エステル型ステロイドを緩徐に常駐
\item 在宅自己注射が可能な喘息bio(最新),bioの適応疾患の一覧
\item 多剤耐性緑膿菌の定義と使える薬について
\item RFP+VRCZ併用禁忌
\item 抗結核薬のうち減感作療法を行うもの(INHとRFP)
\item INHはヒスチジンが蓄積
\item MAC症に対する標準的なレジメンの確認
\item マイクロ波凝固療法は純酸素でも使用可.APCは気道穿孔リスク低い.
\item BTが作用するのは平滑筋細胞(減らす),線維芽細胞(リモデリング改善),迷走神経.

\item G-CSFの一次予防はFN発症率>20\% 
\item コンパニオン診断の使うべき薬
\item CGP
\item 2型肺胞上皮細胞と血管内皮細胞は放射線感受性が高い
\item 肺癌術前の運動負荷試験
\item 百日咳の診断方法
\item 院内肺炎の定義 48時間以降.NHCAPの定義
\item A-DROPの
\item NHCAPまでA-DROPで判定.院内肺炎はI-ROAD.耐性菌リスク因子は過去90日以内のiv ABx,過去90日以内の2日以上の入院歴,免疫抑制薬,活動性低下(PS3以上,BArthel 50未満,歩行不能,経管栄養またはCV)
\item 高齢者における肺炎球菌ワクチンの接種スケジュール
\item 誤嚥性肺炎における嚥下機能評価:簡易嚥下誘発試験のやり方について
\end{itemize}

\begin{itemize}

\item リポイド肺炎,死亡貪食マクロファージ,外因性肺炎,ガソリンや灯油,好物など.4割無症状.TBBで判定.対症療法.
\item マイコプラズマは抗原,LAMP,抗体はPA法,ただしペア血清が必要.培養するならPPLO.IgM抗体は発症7日以降.MLs効かなければTCまたはNQ
\item クラミジア肺炎は潜伏期間3-4w,5類感染症(定点把握),IgM(>10日)/IgGのペア血清で診断.
\item レジオネラはGiemenez染色・アクリジンオレンジ染色.培養はBCYE-αやWYO培地

\item 細菌性肺炎と非定型肺炎の違い
\item CPAの血清診断基準.IPAのCTでは病初期ではhalo sign,回復期ではair crescent signが特徴的である.
\item 肺クリプトコッカス症は届出不要,播種性クリプトコッカス症は5類感染症.Cryptococcus neoformansがほとんどだがC.gattiの報告例もある.治療は基本的にフルコナゾールで播種病変がある場合はAMPH-B+5-FC(レジメンもう一度確認,L-AMBのエビデンスは少ない)
\item ムーコル症の場合はAMPH-B, L-AMB, PSCZなどを用いて,可能であれば外科治療も行う.
\item アゾール系の中で血中濃度測定が必要なものとそうでないもの.

\item Tb: 培養陽性の場合核酸同定検査(DDH)などで判定する
\item Tbでの治療延長を検討する条件(9ヶ月):再治療例,重症例,初期2ヶ月の治療後も培養陽性の場合.
\item INH, RFPに耐性の場合は多剤耐性結核.NQと,(カナマイシン,AML,カプレオマイシンのいずれかに耐性)→超多剤耐性結核.
\item 多剤耐性結核に対するレジメン
\item NTMの臨床診断基準
\item M.absesscusの中でmassilienseはerm41(MLs誘導耐性)がないのでMLs使える.

\item 肺吸虫症は日本では宮崎肺吸虫症とWesterman肺吸虫症がツートップ.気胸,浸潤影,結節,空洞,胸水など.1st choiceはプラジカンデル.
\item トキソカラ症は牛,鶏の生食で感染.アルベンダゾール or メベンダゾール.
\item 糞線虫症は沖縄,奄美に多い.ARDS合併,イベルメクチン.

\item PCPの栄養体は、Wright-Giemsa染色法や、その簡易法であるDiff-Quik法で染色することが可能ですが、嚢胞体は染色できません。メテナミン銀染色やトルイジンブルーO染色は嚢胞体を見つけるのに使用されます。

\item Actinomyces, NocardiaはともにGPR.NocariaはZ-N染色で染色される.Actinomycesは胸壁に病変が進展して瘻孔形成しうる.Nocardiaは脳膿瘍合併.


\item 中皮腫はヒアルロニダーゼ消化試験陽性,微絨毛がみられる,CEA陰性


\end{itemize}
