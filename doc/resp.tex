%!TEX root = ../board-prep.parent.tex


\section{呼吸器(呼吸器専門医試験用)}

\subsection{総論}

\subsubsection{呼吸器の発生と解剖}

\begin{itemize}
\item 前腸(foregut)の腹側に肺芽が出現 → 分岐と伸長を繰り返して肺胞.
\item 胚芽期(26日-6w:気管支)→偽腺管期(6-16w:終末細気管支)→細管期(16-28w:呼吸細気管支,サーファクタント産生開始)→嚢状期(28-36w:間質が減少,サーファクタント分泌完成,胎外生活可能)→肺胞期(36w-).
\item ※なので新生児のRDS予防のサーファクタント投与は<34-36wで行う.
\item 細気管支(φ2mm)の特徴:気管軟骨と気管支腺が消失,club cell>線毛上皮細胞,Millerの二次小葉を支配.
\item Club cell(旧Clala cell)は分裂能があり,CCSP (club cell secretory protein)を分泌する.
\item 細気管支 =(Millerの)二次小葉,終末細気管支 = 細葉(細葉が集まって二次小葉を形成)
\item Millerの二次小葉の本来の定義は「小葉間隔壁によって境された部分」,大きさは0.5cm-3cm.Reidの二次小葉は「複数の細葉が束になった部分」,大きさはほぼ1cmで一定.
\item 呼吸細気管支の定義:壁に肺胞が付着した細気管支.
\item 終末〜呼吸細気管支から,中枢側に向かって逆行する反回枝(娘枝)が出る.
\item 肺胞の直径は0.1-0.2mm.
\item 気管支動脈は,右は肋間動脈,左は胸部大動脈から分岐.
\item 気管支動脈の血流量は心拍出量の1\%.
\item 気管支静脈は奇静脈,半奇静脈から右房に還流する肺外気管支静脈と,肺静脈と吻合しながら左房に還流する肺内気管支静脈の両方が存在する.
\item 胸膜中皮細胞は中胚葉由来,水代謝に関与する.
\item 放射線感受性が高い細胞:2型肺胞上皮,血管内皮.
\item 立位では肺尖部で肺胞気圧>肺動静脈圧のため肺尖部には血液が流れにくい→換気血流比は肺尖部で大きい.
\item 横隔膜の裂孔:①食道裂孔(食道,迷走神経),②大動脈裂孔(大動脈,胸管,奇静脈,交感神経),③大静脈孔(大静脈孔).
\item 横隔膜ヘルニアは①Bochdalek孔ヘルニア(先天性,後外側),②Morgagni孔ヘルニア(前方,胸骨剣状突起の裏側).横隔膜弛緩症(横隔膜の薄くなった部分で,腹腔内臓器が胸腔内に変位する).

\insfig{diaphragm.jpg}{0.4}{https://www.hosp.med.osaka-u.ac.jp/~cfdt/disease/chest/disease-chest04.html}

\end{itemize}


\subsubsection{外科・血管}

\begin{itemize}
\item 開胸する場合,標準的なラインは後側方切開(広背筋→僧帽筋→大菱形筋→前鋸筋→骨)or 腋窩切開,前方切開(胸背動静脈,長胸神経の走行に注意).
\item 気管支断端の離開(気管支断端瘻)は,肺全摘や右下葉切除など太い気管支で起きやすい.
\item 胸壁再建は胸壁合併切除かつ連続3肋骨以上の切除の場合に行う.
\item 通常の肺葉切除で癌がとりきれない場合は気管・気管支形成術を行う.楔状切除(wedge resection) or 管状切除(sleeve resection).吻合部狭窄にはレーザー,バルーン拡張,ステント留置などを行う.
\item 「縦隔リンパ節郭清」を行う場合,系統的リンパ節郭清→ND2a-2まで郭清,選択的リンパ節郭清→ND2a-1(転移頻度の低いリンパ節郭清を省略する,I期肺癌ではND2a-2に非劣性?),
\item 左の上下のが共通幹を形成して左房に還流する例が14\%.肺葉切除で誤って共通幹を切離するとかなりヤバい.
\item みぎV2がV6と一体になり中間気管支幹の背側を走行する破格が6\%,\#7の郭清の際に深追いするとヤバい.
\item 肺癌術後のIP急性増悪のリスクは,術前CTにおけるIP所見,区域切除以上の肺切除,過去のAE-IPの既往,男性,KL-6↑,術前ステロイド使用,\%VC低下(拘束性換気障害).
\item 急性膿胸(<3ヶ月)に対しては胸腔鏡下膿胸空掻爬術,慢性膿胸(>3ヶ月)に至ると膿胸嚢摘除,膿胸腔閉鎖,開窓術(ドレナージ).
\item 横隔神経麻痺はまず呼吸リハビリテーションと夜間NPPV,重症例には横隔膜縫縮術,横隔神経ペーシング.
\end{itemize}


\subsubsection{呼吸生理}

\begin{itemize}
\item ゆっくり肺を広げていくとP-V curveの最初の変曲点がLIP (lower inflection point)で,ここから肺胞が広がり始める.最後の変曲点がUIP (upper inflection point)で,ここを超えると肺が過剰に広がり始める.原理的には人工呼吸器のPEEPをLIP以上,最大気道圧をUIP以下にすることで肺保護換気を達成する.

\insfig{pv_curve.jpg}{0.3}{https://share.google/5lei4ymoPJhpJhvPB}

\item 化学受容器は中枢:延髄腹側(PaCO2),末梢:頸動脈小体,大動脈小体など複数.
\item 低酸素換気抑制:低酸素状態が長く続くと換気量↑のレスポンスが鈍くなる.
\item PaCO2↑に対応する換気量upはPaO2↑で鈍る(要はCO2ナルコーシス),睡眠でも鈍る.
\item 酸素解離曲線の右方シフトは体温↑,アシデミア,2,3-DPG↑.
\item 血管内皮細胞はACEを分泌,ブラジキニンとセロトニンを不活化(分解).
\item アラキドン酸カスケードの起点はcPLA2 (cytosolic phospholipase A2).COX系の脂質メディエーターはPGとTXA2,5-LO系の脂質メディエーターはLT.
\item TXA2は強力な血管収縮・気管収縮(あと血小板凝集).
\item 高地肺水腫の原因は低酸素性の肺血管攣縮,神経原性肺水腫は交感神経活性化による毛細血管圧の上昇.
\end{itemize}


\subsubsection{疫学}

\begin{itemize}
\item 新規の肺癌:男性8万人,女性4万人.
\item 塗抹陽性結核:4例/10万人.
\item 結核死:2000人/年.90歳以上の結核患者の死亡率:50\%.LTBI患者のうち医療従事者:25\%.
\item 喘息死:2000人/年(減少),COPD死:18000人/年.
\item 小児喘息のうち成人喘息への移行率:30\%.
\end{itemize}


\subsubsection{主要徴候と身体所見}

\begin{itemize}
\item Miller \& Jones分類は肉眼所見,P2以上でgood quality(膿性痰1/3以上).
\item Geckler分類は顕微鏡所見,4/5群でgood quality(好中球>25,上皮<25).吸引検体なら上皮<25でありさえすれば好中球が少なくても適正検体としましょう(6群)ということ.

\insfig{miller_geckler.jpg}{}{https://data.medience.co.jp/}
\item ACT:\textbf{20-24点}でコントロール不十分,\textbf{<20点}でコントロール不良.MCIDは3点.小児にはC-ACT.
\item ACTの項目:日常生活への支障,息切れ,夜間の中途覚醒,SABA使用回数,自身での喘息コントロールの自覚.
\item 修正Borgスケール:0〜10,0.5 =「非常に弱い息切れ」.
\item mMRCグレード3:「100mまたは数分歩いて息切れ」.Hugh-Jones分類はほぼmMRCと同じ.

\insdualfig{dyspnea_fhj.jpg}{1}{dyspnea_mmrc.jpg}{1}{https://pulmonary.exblog.jp/19076860/}

\item 嗄声をきたす癌:甲状腺癌>肺癌>食道癌.
\item ばち指:DPD/IPD>1.0(爪甲基部の厚みの方がDIP関節の厚みよりも分厚い = これがばち指の特徴).シャムロス徴候+.指末端でPDGFやVEGFが分泌.
\item 肺性肥大性骨関節症はAd(やSq)に合併.ばち指,四肢長管骨の骨膜新生,関節炎.
\item 抗VGKC抗体:SCLC,胸腺腫.抗VGCC抗体:Lambert-Eaton症候群(つまりSCLC).
\item MGで抗MuSK抗体陽性なら胸腺切除術は非推奨(抗AChR抗体陽性ならDo).
\item Hoover徴候:COPD,振子呼吸:肺結核後遺症.
\item 気管支漏:100mL/日以上の喀痰.
\end{itemize}


\subsubsection{検査}

\begin{itemize}
\item 肺炎球菌は(一応)咽頭ぬぐい液で莢膜細胞壁抗原の検出が可能!(ラピラン®)
\item プリックテストの判定は15分後.H1bは検査4-5日前,H2bは24時間前に中止.LTRAは継続でOK.
\item 皮内テストはプリックテスト陰性例に行う.0.02mLを前腕屈側に注射.15分後に判定.
\item 喀痰細胞診は常温で12時間以内,冷蔵で24時間以内に検査する.(検診ではなく)診断を目的とする場合は1人に対して3回検査する.
\item 被曝量:CXp1回で0.04mSv, 胸部CT1回で7.8mSv,FDG-PETではFDG投与に伴う被曝が3.5mSv,PET-CTでCTを低線量で撮影すると,CT線量は1.4-3.5mSv程度に抑えられる.
\item Lightの基準:胸水TP/血清TP>0.5,胸水LDH/血清LDH>0.6,胸水LDH>2/3*血清ULN.
\item 滲出性胸水の補助診断:①胸水T-Chol>55mg/dL,②胸水T-Chol/血清T-Chol>0.3,③血清Alb-胸水Alb<1.2g/dL.
\item 胸水ADA上昇:リンパ球優位なら結核,リウマチ性胸水.好中球優位なら膿胸,肺炎随伴性胸水.他にリンパ腫やIgG4RDでも上昇.
\item 良性石綿胸水の胸水は好酸球優位.
\item MPMの胸水はヒアルロン酸>10万ng/mL,SMRP上昇(>8-15nmol/L,血清でも上がるがあくまで参考所見).
\item 乳び胸の胸水はリンパ球優位でカイロミクロン上昇.逆に乳び状胸水(chyliform effusion)は胸腔内での炎症の遷延で生じる,カイロミクロンは増加しない.
\item コントラスト心エコー:airを撹拌した生食を末梢Vから投与しながら心エコーをする.普通はRA→RV→PAと行って,微小なairは末梢肺でトラップされてLAには戻ってこない.<3拍でLA〜LVに戻ってくるなら,心内のR→Lシャント(PFO,ASD)が存在する.>3-6拍で戻ってくるなら肺内シャントが存在する(:要するに肺胞と十分に接触しない血管床があるということ.肝肺症候群の診断に有用).
\item DLSTの偽陰性はミノサイクリン(有名),偽陽性はMTX,小柴胡湯.
\item 肺血流シンチでは99mTc-MAA使う(肺の毛細血管にトラップされるので血流がないところが欠損する),肺換気シンチでは放射性ガス(放射性キセノン,クリプトン,Tcなどを用いる)を吸わせて,換気がないところが欠損する.肺血流シンチだけでは偽陽性が多いので,V/Q mismatchの判断には換気血流シンチが重要(CTEPHの診断には換気血流シンチが必須).
\end{itemize}


\subsubsection{PFT}

\begin{itemize}
\item 肺癌手術の目安:術前\%FEV1>50\%,術前\%DLco>50\%.術後予測1秒量>800mL,術後予測\%1秒量>30\%.
\item DLco測定時の混合ガス:N2, O2, He, CO(COは拡散+希釈,Heは希釈をみる).
\item DLcoの注意点:①smokerは低めに出る(検査前24時間は喫煙),②食後2時間,運動直後を避ける.
\item DLco/VA:ガス交換面積あたりのDLcoを見る → COPD(気腫型)ではVA大きいのでDLcoより低く出る,IPではDLcoより高めに出る.
\item アストグラフ:BAの気道過敏性試験.メサコリン吸入.
\item V50/V25:それぞれ肺気量の50\%(25\%)のときの呼気速度.>3.5で末梢気道閉塞.
\item CV (closing volume):末梢気道閉塞を見る.100\% O2を吸うと肺底部に溜まって肺尖部は空気(N2)が相対的に多い→息を吐くとまず下肺野のO2が先に出る→さらに息を吐いてN2が増え始めたところで下肺野は空気が抜けて(or 末梢気道病変のために)つぶれたと考える.\item CVは普通10\%,20-25\%を超えると末梢気道閉塞.
\item 炭酸ガス換気応答テスト (VR-CO2):呼気を繰り返し吸わせてCO2↑→普通は換気量up.先天性中枢性低換気症候群 (CCHS) では換気量upの反応がなくなる.
\item 広域周波オシレーション法:音響スピーカーなどによる工学的な空気振動を安静呼吸で吸ってもらいPCで解析.呼吸インピーダンス (respiratory system impedance; Zrs) を測定し,これを呼吸抵抗 (respiratory system resistance; Rrs),呼吸リアクタンス (respiratory system reactance; Xrs)に分解する.
\item BA, COPDではRrsが増加し,特に低周波数領域でRrsが高くなる.
\end{itemize}


\subsubsection{BFS}

\begin{itemize}
\item BALの正常所見:細胞数〜13万,Mph 90\%,Ly 10-15\%,CD4/8 1-2.
\item BALのCD4/8比↑:サルコイドーシス,結核,ベリリウム肺,慢性HP.CD4/8比↓:CTD-ILD(NSIP, COP),急性〜亜急性HP,珪肺??
\item 100\% O2投与下で使用可能なのはマイクロ波凝固療法とクライオ.APCは気道穿孔リスク低い.
\item BTが作用するのは平滑筋細胞(減らす),線維芽細胞(リモデリング改善),迷走神経.
\end{itemize}


\subsubsection{治療総論}

\begin{itemize}
\item 去痰薬は「キレを良くする」ならアンブロキソール,アセチルシステイン,「量を減らしたい」ならカルボシステイン,フドステイン(クリアナール®),「量が増えてもしっかり出したい」ならビソルボン.つまり漿液性痰にビソルボンは使えない.
\item 14員環マクロライドはEM, CAM, ルリッド®,15員環マクロライドはAZM(16員環マクロライドは本邦では使えない).
\item 肺胞出血に対して血漿交換が適応になるのは,①Goodpasture,②ANCA-RD,③SLE(特にRPGNまたはCNSループス合併例).
\item 肺胞出血の薬剤:抗癌剤(MTX,マイトマイシンC,CyA),抗てんかん薬(フェニトイン,カルバマゼピン),抗不整脈薬(アミオダロン,キニジン).
\item AZA→NUDT15遺伝子コドン139の多型が保険適応(白血球減少,脱毛が出やすい).
\item CPT→UGT1A1遺伝子多型(下痢が出やすい)
\end{itemize}


\subsubsection{医療費制度}

\begin{itemize}
\item 指定難病による医療費助成:指定難病のうち重症度が一定以上の場合,自己負担が基本2割となり,さらに1ヶ月あたりの負担上限額が設定される.
\item ただし,指定難病で医療費助成を受けられない「軽症者」であっても,月ごとの医療費総額(10割負担額)が33,330円を超える月が、年3回以上ある場合は医療費助成の対象となる(いわゆる軽症高額).
\item 身体障害者手帳:障害者控除(税金),交通機関の割引運賃などの福祉サービス.心身障害者医療費助成としてさらに医療費の負担が減る可能性も.
\item 身体障害者手帳(呼吸器機能障害)の判断項目は,①指数(予測肺活量1秒率:つまり1秒量/予測肺活量,FEV1でも\%FEV1でもない!!!)②PaO2.
\item 障害者手帳1級:指数≦20 or PaO2≦50Torr,3級:指数20-30 or PaO2 50-60,4級:指数30-40 or PaO2 60-70.
\item 「お題目」としては,手帳1級がmMRC 4,3級がmMRC 3,4級がmMRC 1-2.
\end{itemize}

\subsection{疾患}

\subsubsection{SAS}

\begin{itemize}
\item CPAPは日本で50万人.
\item エプワース眠気尺度:>11点で高リスク → 簡易PSG(アプノモニター)に進む.

\insfig{epworth_scale}{0.6}{https://hirai-naika-geka.com}

\item 簡易PSGでは純睡眠時間は分からないので,代わりにモニター装着時間を使う → AHIではなくREI (respiratory event index).モニター装着時間の方が長いのでREIはAHIより低く出る.
\item PSGの記録チャンネル7個:脳波,眼電図,筋電図,心電図,気流,呼吸努力,SpO2.

\insdualfig{psg.jpg}{1}{sas_diagnosis.jpg}{0.8}{https://www.chibatoku.or.jp/, http://www.take-clinic.com/psm/sas.htm}

\item OSA重症度(AHI):軽症5-15,中等症15-30,重症>30.
\item CPAP保険適応基準:アプノモニターでAHI>40,PSGでAHI>20.
\item CPAPの基本モードはauto CPAP(気道閉塞に応じてCPAPの圧を自動に調節,普通は5-15cmH2Oくらい),ただし理想はPSGをやりながら至適圧を調整するマニュアルタイトレーションをやりたい.
\item CPAPの目標は≧4時間/日.
\item CPAP使用しても日中の眠気が続く場合はモダフィニル使用可.
\item CPAPがどうしても継続できない場合に舌下神経電気刺激療法:右前胸部の皮下にパルスジェネレータを植え込み,舌下神経に電気刺激を与えて,舌根沈下を防ぐ.最初は全麻で植え込み手術が必要.肥満(BMI>30)は適応外.
\item CSAでは呼吸調節システム自体が不安定になる(心不全など).CPAP装着下でAHI>15ならASV(adaptive servo ventilation)を検討する.
\end{itemize}


\subsubsection{細菌感染症}

\begin{itemize}
\item A-DROPはSpO2<90\% (r/a),I-ROADはSpO2<90\% (FiO2 35\%).
\item CURB65:年齢>65,呼吸数>30(SpO2ではない).残りはA-DROPと同じ.
\item NHCAP:①施設,②<90日以内に退院,③要介護,④定期的な血管内治療(透析,ケモ,免疫抑制薬,ABx).
\item NHCAPもA-DROP使う.
\item 耐性菌リスク:①<90日の静注ABx投与歴,②<90日の2日以上の入院歴,③免疫抑制薬,④活動性低下(PS≥3,BI<50,歩行不能,経管栄養,CV)

\insdualfig{adrop.jpg}{1}{iroad.jpg}{0.9}{https://knowledge.nurse-senka.jp}

\item 定型肺炎と非定型肺炎の鑑別(諸外国には浸透していない):①<60歳,②基礎疾患なし,③ひどい咳,④聴診でcrackleなし,⑤喀痰から起因菌生えない,⑥WBC<10000.3/5または4/6以上で非定型疑い.
\item 多剤耐性緑膿菌(MDRP)の定義:カルバペネム,アミノグリコシド,キノロンに耐性.
\item 耐性グラム陰性桿菌(CREなど)の治療薬:コリスチン,チゲサイクリン,セフトロザン・タゾバクタム(ザバクサ®),セフタジジム・アビバクタム(ザビセフタ®),セフィデロコル(フェトロージャ®),イミペネム・シラスタチン・レレバクタム(レカルブリオ®).ザバクサ,ザビセフタが耐性でもフェトロージャ,レカルブリオは感受性の可能性あり.
\item 肺炎球菌ワクチン:経過措置は終了し,現在の定期接種は「>65歳」or「60-65歳の重症患者」にPPSV23を1回.
\item PPSV23の代わりにPCV20,PCV21,PCV15→PPSV23(1年以上あけて)の連続接種はありだが,いずれも自費扱い.

\insfig{pneumonitis_vaccine.jpg}{}{65歳以上の成人に対する肺炎球菌ワクチン接種に関する考え方(第7版)}

\item 誤嚥性肺炎に対する嚥下機能評価:①反復唾液嚥下テスト(RSST:30秒で唾液3回以上飲めればOK),②簡易嚥下誘発試験(SSPT:中咽頭までカテーテル入れて0.4mLの5\% Gluを注入,3秒以内に嚥下すればOK),③水飲み試験(WST:普通に水飲んでもらう).
\end{itemize}


\subsubsection{百日咳}

\begin{itemize}
\item 血清抗体は発症2週〜.PCR/LAMPは〜発症3週で採取.抗原検査(鼻咽頭)も存在するが,偽陽性が多い.
\item 抗原検査キット(イムノクロマト法)も保険収載されている.
\item PT-IgGはワクチンの影響を受けるので基本的に単血清では診断できない.ただし>100 EU/mLなら新規感染.
\item PT-IgGでペア血清をとる場合,「1回目<10 EU/mL,2回目>10 EU/mL」or「1回目 10-100 EU/mL,2回目で1回目の2倍以上」ならOK.
\item IgM/IgAはワクチンの影響を受けないが陽性率低い.IgMは発症2週,IgAは発症3週でピーク.
\item 発症3-6w経つとABxは症状改善には寄与しないが,二次感染を防ぐ意味でMLs投与してよい.
\end{itemize}


\subsubsection{非定型肺炎}

\begin{itemize}
\item マイコプラズマの診断は抗原 or LAMP法を優先する.抗体(PA法ただしペア血清,4倍以上の上昇).培養するならPPLO培地が必要.
\item PA法は主にIgM(と少しIgG)を測る.day7から上昇.MLs開始して2-3日で解熱しなければテトラサイクリン or ニューキノロンに変更.
\item クラミジア肺炎は潜伏期間3-4週,5類感染症(定点把握),IgM(>10日)やIgGのペア血清で診断する.テトラサイクリン or MLs.
\item レジオネラの観察にはGiemenez染色,アクリジンオレンジ染色.培養はBCYE-α培地,WYO培地を使う.
\item Q熱はウシ,ヒツジなど反芻動物の尿,糞,乳汁から感染しヒトヒト感染は稀.急性型は全身症状でインフルエンザ様.抗カルジオリピン抗体IgG抗体陽性.
\item Q熱で急性症状を発症するのは40\%で残りは無症候性感染となるが,1-2\%は心内膜炎などを呈して慢性型に移る.急性型は2-3週,慢性型は2-3年(場合によっては一生!!!)のテトラサイクリン投与.
\item Q熱の原因はCoxiella brunettiはLCV (large-cell variant) とSCV (small-cell variant) があり,SCVは芽胞(要は休眠型).牛乳で40ヶ月生存.
\end{itemize}


\subsubsection{結核}

\begin{itemize}
\item IGRAの偽陽性はM.kansasii(有名), M.szulgai, M.marinum, M.gordonae.
\item (旧)ガフキーは1〜10号で判定.現在は使わない.1+はG2(数視野に1個),2+はG5(1視野に4-6個),3+はG9(1視野に51-100個).
\item 培養は液体培地(MGIT法)で6週間,固形培地(小川培地)で8週間.INH耐性は液体培地だと偽陽性になりやすい(固形培地までやるとSにひっくり返ることがある).
\item 菌の同定は,培養前の検体(喀痰など)に対してPCR(TB,avium, intracellulareが検出可能),培養陽性になった菌株に対して質量分析法(159菌種).DDHマイコバクテリア法は受託中止(小川培地から生えた菌しか検査できず,しかも18種類しか検査できなかった).
\item A法:2HREZ+4HR,B法:2HRE+7HR.
\item A法,B法で+3HRを追加する条件は,①治療開始2ヶ月以降で培養陽性,②空洞形成など重症結核,③再治療,④免疫低下 ※結核性胸膜炎での治療延長は不要 ※透析患者も延長不要
\item 腎障害がある場合はEBとPZAを週3回に減量(実際にはB法でHR連日+EB週3回にする).
\item RFPとINHは減感作療法が可能.いったん両方休薬して1剤ずつ減感作プロトコルに従ってdose upする.
\item RFP+VRCZ併用禁忌.INHはヒスチジンが蓄積して発疹出やすい.
\item PZAの禁忌は妊婦,(ADL不良)高齢者.PZA眼障害のリスクは腎不全,DM,肝障害,低栄養,喫煙,飲酒.
\item 結核の初期悪化:CXpで陰影悪化,胸水,リンパ節腫脹.あくまで効果判定は喀痰培養で判断!抗結核薬は継続で良い!
\item 治療中断した場合:初期治療時は14日以上中断したら最初からやり直し.そうでなければ継続.維持期では80\%以上の服薬が終了していれば継続(最後まで飲み切る).80\%未満で止まっていれば,「中断合計期間が3ヶ月以上 or 連続での中断が2ヶ月以上」の場合は治療強化期(つまりHREZ)からやり直し.
\end{itemize}


\subsubsection{耐性結核}

\begin{itemize}
\item INH耐性結核にはINHの代わりにLVFX.GLの推奨は6RELZ (RFP+EB+LVFX+PZA)で,軽症例に限っては2RELZ+4RELも許容範囲.PZAが一切使えない場合は12REL.
\item ※RFP耐性結核にはRFPの代わりにLVFX.推奨は①6HELZ+12HEL,②6HESL (INH+EB+SM+LVFX)+12HEL.菌陰性化から18ヶ月.
\item 多剤耐性結核の場合,①キードラッグはLVFXとBDQ,②なるべくLZD,③EB, PZA, DLM(デラマニド), CFZ(クロファジミン), CS(サイクロセリン)の中から選んで,合計5剤使う.基本は菌陰性化から18ヶ月.
\item 超多剤耐性結核→NQ耐性かつKM,AMK,カプレオマイシンのいずれかに耐性.
\end{itemize}


\subsubsection{LTBI}

\begin{itemize}
\item LTBIの治療レジメン:4HR or 6H,INHが使用できない場合に限り4R(RFP投与後のRFP耐性結核の発症リスクが上昇しないというエビデンスが無いため1stにはならない).
\item LTBIの治療適応は下記を参照だが,①接触者検診で陽性は治療,②DMはコントロール不良の場合のみ治療,③PSL>15mgを>1ヶ月は治療.
\item RAはデフォルトでLTBIの高リスクで,TNF-α阻害薬でさらにリスク上昇.TNF-α阻害薬を使う場合,3週前からLTBI治療スタート.

\insfig{ltbi.jpg}{}{潜在性結核感染症治療指針}

\end{itemize}


\subsubsection{HIV+結核}

\begin{itemize}
\item HIV+LTBIは抗結核薬がmust.ただし検査感度の問題からCD4>200になってからT-SPOTを出す.
\item 未治療HIVで結核が判明したら,まず結核治療を先行して,2週後をめどに抗HIV治療を開始する.※HIVだからと言って治療延長は不要(ただし,若干のcontroversyはあり)
\item ただしHIV治療開始によるIRISで結核が悪化しうる.重症例はOCS使って良い.※やはり治療効果判定は喀痰培養で
\item RFPは抗HIV薬との相互作用が多いのでなるべくRFB(リファブチン)を使う.

\end{itemize}

\subsubsection{NTM}

\begin{itemize}
\item 迅速発育菌はabscessus, Chelonae, Fortuitum.
\item NTMの診断基準の基本は臨床診断(CT所見)+細菌学的診断(喀痰2回培養陽性,BAL/気管支洗浄液で1回陽性,喀痰1回陽性+肺病理がcompatible).
\item 2024年のNTM暫定的診断基準では,細菌学的診断として(喀痰2回陽性の代わりに)「喀痰1回陽性+キャピリアMAC抗体陽性」or「喀痰1回陽性+胃液1回陽性」でもOK.
\item 感受性検査は,MACではCAMとAMK,kansasiiではCAMとRFPを確認.
\item MAC症に対する治療は,CAMの代わりにAZM (250mg/d)使用可.NB型なら週3回投与でもOK.FC型/重症NB型では開始3-6ヶ月はSM/AMKを併用.
\item RECAMを6ヶ月継続しても治癒しない(←厳密な定義はない)難治例はRECAM継続で,SM/AMK/アリケイス® 追加する.
\item M.kansasii → 歴史的にはHREが行われていたがRECAMの奏効率も高い.
\item M.abscessus → IMP/CS+AMK+クロファジミン(ランプレン®)+MLs → ランプレン+MLs(MLs耐性ならMLsの代わりにIMP/CS or AMKはなるべく継続).
\item M.absesscusの中でmassilienseはerm41がないのでMLs誘導耐性を起こさない.ただしabscessusの中でerm41を有するが,特にC28 sequevarである場合,例外的にMLs誘導耐性を起こさずMLs使える.
\end{itemize}


\subsubsection{その他の細菌}

\begin{itemize}
\item Actinomyces: GPR,胸壁に病変が進展して瘻孔形成しうる.1st choiceはペニシリン系.
\item ノカルジア:GPR,Ziehl-Neelsen染色で染色される.脳膿瘍リスク.1st choiceはTMP-SMX.
\item 肺吸虫症は日本では宮崎肺吸虫症(イノシシ食べる)とWesterman肺吸虫症がツートップ.腹腔→胸腔(胸水)→気胸,浸潤影,結節,空洞影.プラジカンデル.
\item 肺吸虫症で「大網,腸間膜,腹膜などに病変を認めた場合には,結核性腹膜炎や悪性腫瘍の腹膜播種との鑑別が難しい」,「慢性的な咳,血痰・喀血・空洞形成」で肺結核のmimicにもなる!末梢血好酸球増多があれば肺吸虫症を疑う.
\item トキソカラ症は牛,鶏の生食で感染.アルベンダゾール or メベンダゾール.
\item 糞線虫症は沖縄,奄美に多い.ARDS合併する,イベルメクチン.
\end{itemize}


\subsubsection{PCP}

\begin{itemize}
\item PCPは栄養体であればGiemsa染色,Diff-Quik法で染色可能だが,嚢胞体は染まらない.嚢胞体を染めるにはメテナミン銀染色,トルイジンブルーO染色が必要.
\item HIV-PCPで上葉の胸膜直下に嚢胞きたしうる.
\item 軽症〜中等症では1st TMP-SMX, 2nd アトバコン(サムチレール®),中等症〜重症では1st TMP-SMX,2nd ペンタミジン点滴.治療期間はHIV-PCPで3週,non HIV-PCPで2週.
\item ペンタミジン(ベナンバックス)を使う場合,PCP治療では静注のみ,予防では静注 or 吸入(いずれも月1回程度,吸入は効果落ちる).
\end{itemize}


\subsubsection{肺真菌症}

\begin{itemize}
\item FLCZはカンジダ+クリプト(しかもC.kruseiとC.glabrataには無効→これらにはMCFG使う).ITCZはアスペルギルス有効だがbioavailability超低い.
\item VRCZはTDM必要,副作用多い→眼,肝,皮膚(光線過敏,紅斑).
\item ISCZはCPA+IPA両方の適応があり,TDM不要で,サイクロデキストリン入っていないので腎障害あってもdiv可能.
\item L-AMBは真菌なんでも有効かと思いきや,アスペルギルスのうちterreusとflavusは効きにくい!
\item アスペルギルス抗原 = アスペルギルスガラクトマンナン抗原 = GM抗原.感度・特異度は80\%程度.BALFでアスペルギルス抗原測定するとIPAの感度が上昇する.他の糸状菌(フサリウム,ヒストプラズマetc)で偽陽性.
\item IPAの1stはVRCZ,2ndはISCZ, L-AMB,PSCZ.
\item IPAのCTでは病初期ではhalo sign,回復期ではair crescent sign.EORTC/MSGの診断基準では病理学的に組織障害+アスペルギルスが証明されれば確定(proven).
\item CPAの1stはVRCZまたはキャンディン系(MCFG, CPFG),2ndはISCZ, L-AMB(PSCZはVRCZに非劣性示せず保険適応がない).

\item ABPM診断基準.現在はアスペルギルス沈降抗体は受託中止,アスペルギルスIgG抗体を使用する.

\insfig{abpm.png}{}{https://pulmonary.exblog.jp/30457004/}

\item ABPMの原因真菌はアスペルギルス,ペニシリウム属,スエヒロタケ(Scizophyllum commune).

\item 肺クリプトコッカス症は届出不要,播種性クリプトコッカス症は5類感染症.Cryptococcus neoformansがほとんどだがC.gattiの報告例もある.
\item クリプトコッカスはBDG上昇しにくい(播種性クリプトコッカスでは上昇する,ちなみに接合菌も上昇しない).
\item クリプトコッカス抗原 = グルクロノキシロマンナン (GXM)抗原.ただし播種性トリコスポロン症でも上昇する.
\item 肺クリプトコッカス症のGXM抗原の感度・特異度はともに90\%以上(テキスト)とされるが,実際はおそらくもっと低い,HIV陽性例のクリプトコッカス髄膜炎の感度は高い.
\item 肺クリプトコッカスの治療はFLCZを3ヶ月,播種病変がある場合はL-AMB+5-FCを最低2週,その後FLCZによる地固め療法(400mg8週)→維持治療(200mg).
\item クリプトコッカスにはキャンディン系は無効!
\item ムーコル症の1st choiceはL-AMBで,代替薬はPSCZ,ISCZ.可能であれば外科治療も併用.
\item トリコスポロンは免疫不全者に播種性トリコスポロンとして発症しうる!キャンディン系投与中のブレイクスルー感染としても鑑別にあげる!
\end{itemize}


\subsubsection{BA}

\begin{itemize}
\item LTC4, TGF-βで気道平滑筋が肥厚する.PGE2は例外的に気道拡張.
\item Th2細胞→IL-4(B細胞をクラススイッチさせてIgE産生,気道粘液),IL-5(好酸球増殖),IL-13(気道過敏性,IgE産生). 
\item 好酸球からMBP (Major basic protein)分泌→気道過敏性亢進.
\item IL-25,IL-33,TSLP → ILC2(細胞表面にIL-4Rが発現,ステロイド抵抗性) → IL-5, IL-13.
\item 粘液細胞からMUC5B, 杯細胞からMUC5ACが分泌されて気道上皮の表層(ゲル層)を構成.MUC5ACがBA悪化と関連(DPBでも増加する).
\item ダニとハウスダストのIgEは95\%以上かぶる.舌下免疫療法の適応は5歳以上.
\item AERDは1:2で女性に多い,20-40歳.IgE介在しない.
\item ピークフロー(PEF):最大呼気流速を測定.日内変動の正常上限は<8\%.コントロールの目標は日内変動<予測値の20\%.
\item 気道可逆性検査:FEV1測定後,SABAを吸入し,15-30分後に再度FEV1を測定.改善率>12\% and 改善量≥200mL で気道可逆性あり.
\item BAの臨床的寛解の基準:3コンポーネントなら①ACT≥23,②増悪なし,③OCS連用投与なし.4コンポーネントなら①②③+④「気管支拡張薬使用後,\%FEV1≥80\% で,かつ\%FEV1やPEFの日内変動<10\%」.
\item ※(教科書的には)LABA単剤治療は気道炎症を悪化
\item テオフィリンの血中濃度は5-15μg/mL,消化性潰瘍.
\item Bio適応疾患:慢性蕁麻疹はゾレア+デュピクセント(結節性痒疹はデュピクセントonly),慢性副鼻腔炎はヌーカラ+デュピクセント,EGPAはヌーカラ+ファセンラ.
\item ゾレアとデュピクセントは好酸球増多の副作用あり.EGPA発症の報告もある.
\item ファセンラはEGPAなら自己注射OK(BAは適応なし)!
\item depemokimab(抗IL-5抗体,6ヶ月に1回投与)は今後市場に出る.
\insfig{ba_bio.jpg}{0.7}{https://medical.nikkeibp.co.jp/leaf/mem/pub/report/202506/589032.html}

\end{itemize}

\subsubsection{COPD}

\begin{itemize}
\item COPDの疾患関連遺伝子としてSERPINA(AATDの原因遺伝子).
\item COPDに関連するサイトカインとしてIL-17A,IL-1β,IL-6,TNF-α.
\item COPD患者ではMMP↑,プロテアーゼ活性↑,好中球エラスターゼ↑.Tc1型CD8+T↑,Th1型CD4+↑,Th17型CD4+↑.
\item Brinkman Index = pack-years * 20.
\item COPDの病期分類(GOLD分類):FEV1\%<70\%の条件のもとで,I期(\%FEV1≧80\%),II期(50-80\%),III期(30-50\%),IV期(<30\%).
\item CATスコアは軽症<10点,中等症10-20点,重症21-30点,超重症>30点.

\insfig{cat.jpg}{0.5}{https://www.gold-jac.jp/}

\item COPDにおけるNPPV導入基準:①動脈血pH≦7.35かつPaCO2≧45 Torr,②呼吸筋疲労,呼吸仕事量増大を伴う呼吸困難.
\item COPDで在宅NFHC可能.HOTを導入していて①PaCO2 35-45,②PaCO2≧45だがNPPVの認容性がない,③夜間に低換気による低酸素がある(これの証明にはPSG or 終夜SpO2測定が必要).
\item ※ただし,在宅HFNCのFiO2は35-40\%程度しか出ない.
\item COPDに対する新規治療薬:エンシフェントリン(PDE3/4阻害,吸入),ロフルミラスト(PDE4阻害,経口).※同じPDE4阻害薬のネランドミラストはIPFの治療薬(本邦未承認)
\item COPDに対するデュピクセントの適応は,「ICS+LABA+LAMAで増悪があり,Eosino≧300/μL」.
\item COPD患者へのMLs (CAM, AZM)は増悪を抑制する,ただし保険適応外.
\item COPD患者に対するワクチン接種の推奨:インフルエンザ,肺炎球菌(5年おき接種が推奨),RS(CDCでは慢性呼吸器疾患があれば推奨),シングリックス(全COPD患者に推奨).
\item 重症例には肺容積減量手術(<80歳,病変が上葉に限局),気管支バルブ(側副換気がない場合).
\item AATは常染色体劣性遺伝,AATインヒビター(リンスバッド®)補充療法が保険適応.
\item ACOはGOLDでは病名として定義されておらず,COPDのフェノタイプとしてCOPD-A(COPDと喘息合併)を定義している.
\item ※肺嚢胞と肺気腫は別.肺嚢胞は喫煙と関連しない.SjDに合併.スキューバダイビングは制限.

\end{itemize}

\subsubsection{BE}

\begin{itemize}
\item BEの背景はRA, SjD, GERD, PCD, CFなど.
\item PCDは常染色体劣性遺伝,DRCA1遺伝子欠失が多いが原因遺伝子は多様.スクリーニングはサッカリンテスト(人工甘味料を下鼻甲介に置いて,味を感じるまでの時間の遅延),鼻腔内NO異常低値.
\item BEのCT:気管支内腔が隣接PAの1.5倍以上(ここは諸説ある).
\item BEの重症度評価(予後因子):FACED score,BSI.
\item BEの予後不良因子:Pseudomonas検出.\%FEV1<50\%,mMRC≧2.

\insfig{be_prognosis.jpg}{}{呼吸器ジャーナル Vol.72 no.2 2024}

\item non-CFのBEに対してDPP-1阻害薬(ブレンソカチブ)がFDA承認.
\item DPB:HLA-B54, 典型例では慢性副鼻腔炎を合併するので耳鼻科受診.好中球性炎症で,MUC5AC増加(ムチンのコア蛋白:BA増悪因子でもある).寒冷凝集素価上昇(>64倍),増悪は肺炎球菌やHibが多い.
\item CF:CFTR遺伝子変異,汗の塩化物イオン濃度上昇.合併症:膵臓の外分泌機能不全,慢性副鼻腔炎,肝硬変,両側精管欠損.CFによるBEに対してのみトブラマイシン吸入が保険適応.
\end{itemize}


\subsubsection{BO/BOS}

\begin{itemize}
\item 環境因子:NO2, SO2, アスベスト吸入,薬剤性:アマメシバ,D-ペニシラミン,感染症:RS, アデノ,マイコプラズマ,2次性:SLE,RA,悪性リンパ腫,移植.
\item 造血幹細胞移植後の慢性GVHDの一病型として出現しうる(中央値14ヶ月,5-10\%,末梢血移植の方が多い),急性GVHDあるとリスク上昇.→なので移植後は定期的にPFT必要.
\item 臓器移植後にBO起こる,特に肺移植後にもよく起こる(CLAD: 移植片慢性機能不全,50-60\%).BALでNeu主体の細胞数増多.
\item BOにステロイドは無効!根治治療は肺移植のみ(矛盾してるけど).
\item BOのCTでは肺野透過性は亢進.Air trappingの検出のため呼気CT撮る.中枢気道の気管支拡張は進行期まで起こらない.
\item 病理はconstructive bronchiolitis,病変間に正常気道が介在する.
\end{itemize}


\subsubsection{IP}

\begin{itemize}
\item 指定難病としての「IIPs」に含まれる疾患は,IPF, NSIP, RB-ILD, DIP, COP, AIP, LIP, PPFE, UCIP.
\item IIPsの指定難病の医療費助成対象は重症度III度以上.
\item 6MWTでmin SpO2<90\% まで下がれば無条件でIII度 = 医療費助成.
\insfig{ipf_severity.jpg}{}{https://pro.boehringer-ingelheim.com/jp/product/ofev/update-of-diagnostic-criteria-for-iips}
\item IPの最新の分類(ERS/ATS 2025)での変更点は,

① IIPsとsecondary IPをまとめて分類,

② まず間質性パターン(interstitial)と肺胞充填性パターン(alveolar filling)に分類し,間質性パターンをさらに線維化と非線維化に大別,

③ 間質性・線維化パターンの病型としてUIP,NSIPと細気管支中心性間質性肺炎(bronchiolocentric interstitial pneumonia, BIP)の3つに分類,

④ BIPは名前の通り細気管支を主体とした病変で,HP,CTD-ILD,DILDなどを包括的に含む,

⑤ AIP→特発性DADに名称変更,

⑥ DIP→肺胞マクロファージ肺炎(alveolar macrophage pneumonia, AMP)に変更.肺胞腔内のマクロファージの充満,喫煙と関連,

⑦ 肺胞充填性パターンの病型としてOP,RB-ILD,AMP,その他(AEP,CEP,PAP,リポイド肺炎),

⑧ MDD診断で確信度≧90\%(condifent),51-89\%(provinsional),≦50\% (unclassifiable)に分類.

\item ILDのGAPインデックスは①IP分類(UIP/UCIPは予後不良),②年齢,③性別,④呼吸機能(\% FVCと\% DLco)で,日本人向け修正GAPモデルでは②③④を使う(重み付けも変更).

\item PFDの副作用は光線過敏,NTDの副作用は下痢,血栓塞栓症.いずれも肝障害では使用不可.
\item IPFに対する新規治療薬:PDE4阻害薬(ネランドミラスト).※同じPDE4阻害薬であるロフルミラストはCOPD治療薬(本邦未承認)
\item PPFは①症状,②PFT,③画像のうち2つ以上が悪化していれば診断可.②は「1年以内に\%FVCが5\%以上低下」または「\%DLcoが10\%以上低下」\footnote{GAPモデルと同一のパラメータ.}.PPFに対してはNTDのみ適応あり.

\item AE-IPFの定義は「1ヶ月以内」「症状・画像・ΔPaO2>10mmHgがすべて揃う」「感染症・気胸・癌・心不全・PEでない」.

\end{itemize}

\subsubsection{PPFE}
\begin{itemize}
\item PPFE:薬剤性(CYが有名),悪性腫瘍(MDS),GVHD,肺移植後.
\item 特発性PPFEの多くは非喫煙者!!るいそう,扁平胸郭,気胸をきたす.ばち指とfine cracklesは比較的レア.
\item CXpで両側肺門挙上,残気率上昇とBMI低値の組み合わせで診断!
\item PPFEの病理は胸膜下弾性線維増生(subpleural elastosis),肺胞内線維化(intra-alveolar collagenosis),臓側胸膜の線維性肥厚(pleural thickening).
\end{itemize}


\subsubsection{CTD-ILD}
\begin{itemize}
\item CTD-ILDに対する免疫抑制は,基本的にMMF,AZA,RTX,CYで選ぶ.SScとMCTDはアクテムラ®使ってよい.RAにはJAKi使ってよい.
\item IPAFの臨床ドメインは手指潰瘍,メカニックハンド,関節炎,手掌血管拡張,Raynaud,末梢性浮腫,Gottron徴候.
\item IPAFの形態ドメインにおいて,CTパターンはNSIP, OP, LIPのいずれかが必要(UIP/pでIPAF診断は不可).
\item RA,SLEではBOあり得る,SjDは悪性リンパ腫(特にMALToma)のリスク.
\end{itemize}


\subsubsection{ARDS}
\begin{itemize}
\item ARDSのベルリン定義(旧):侵襲または呼吸器症状出現から1週以内,両側性の陰影,PEEP≧5cmH2OでP/F<300(<200で中等症,<100で重症).小児では片側陰影でも診断可.
\item 新グローバル基準:挿管してなければ「PEEP≧5cmH2O or HFNC≧30LでP/F≦300」,挿管していれば「P/F>200: mild,P/F<200: moderate,P/F<100: severe」
\item ARDSは予測体重ベースで6-8mL/kgの換気.筋弛緩は<48時間で終了.好中球エラスターゼ阻害薬は一応保険収載.
\end{itemize}


\subsubsection{LAM}

\begin{itemize}
\item 30-40代女性に好発,ただし「女性にしか発症しない」わけではない.肺の多発嚢胞→気胸を繰り返す.
\item 腎血管筋脂肪腫は,孤発例の30\%,TSC例の80\%に合併する.
\item ※LAMで血清VEGF-D高値(≧800pg/mL).CT所見+VEGF-D高値でLAMと診断できる.
\item LAM細胞のクラスターは乳び胸水・腹水にみられる.LAM細胞はD2-40,α-SMA,HMB45,ER/PR陽性→低容量ピルでLAM悪化するかも.
\item mTOR阻害薬として,まずLAMに対してシロリムス(ラパリムス®),TSCに対してはエベロリムス(アフィニトール®).従ってTSC-LAMには両薬剤投与可.※タクロリムスではない
\item 妊娠はLAMの進行を早める.mTOR阻害薬は妊娠12週前には中止が必要.GnRHによる偽閉経療法を検討してもいいが保険適応外.
\end{itemize}


\subsubsection{PAP}

\begin{itemize}
\item 抗GM-CSF抗体陽性であれば自己免疫性PAP.陰性であれば先天性PAPまたは続発性PAP(分類不能の場合は特発性PAP).
\item ※先天性PAPの原因としてSP-B,SP-C,ABCA3遺伝子異常,GM-CSF受容体異常,GATA2異常(先天性PAPの場合は抗GM-CSF抗体は関係ない).
\item 続発性PAPの原疾患:血液疾患(MDS,CML,悪性リンパ腫),粉塵吸入,アミロイドーシス,臓器移植後など.HLA-DRB1遺伝子が発症と関連.
\item BAL液中に泡沫状マクロファージ,末梢気腔内にPAS染色,SP-A染色陽性の顆粒状の無構造物質.
\item PAPの20\%は自然寛解.治療は全肺洗浄,自己免疫性PAPに対してのみGM-CSF吸入(サルグラモスチム,サルグマリン®).難治例にはリツキシマブ or PE(国内では保険償還なし).ステロイドは使用不可!
\item MDS-PAPに対しては造血幹細胞移植を検討してもよい(エビデンスは弱い,そもそも呼吸状態が悪ければ適応自体が難しいが).
\end{itemize}


\subsubsection{ANCA-AAV}
\begin{itemize}
\item プロピルチオウラシル(PTU)でMPO-ANCA陽性AAV.
\item GPAの病理はpalisading granuloma:中心部の壊死に対して柵状に組織球と巨細胞が取り囲む.
\item GPA/MPAの寛解導入はステロイドに免疫抑制薬を追加.1st-choiceはCY or RTX,これらが使用不可ならMTX or MMF.免疫抑制薬併用下で補体C5aの受容体拮抗薬(アバコパン,タブネオス®)は高用量ステロイドよりも推奨度が高い.寛解維持はステロイド+RTX(またはAZA)で行う.
\item EGPAの寛解導入はステロイド+IVCY,寛解維持療法は基本的にステロイド単剤で重症例はステロイド+MTX,ステロイド+AZA.効果不十分の場合はヌーカラ® or ファセンラ®併用.
\end{itemize}


\subsubsection{抗GBM病,Goodpasture症候群}
\begin{itemize}
\item Goodpasture症候群はRPGNと肺胞出血により特徴づけられる.疾患の総称としては抗GBM病だが,90\%はRPGN,25-60\%は肺胞出血.
\item 抗GBM病の30\%でANCA陽性,逆にANCA-AAVの5\%で抗GBM抗体陽性.
\item Goodpasture症候群に対して,治療はステロイド+免疫抑制薬+血漿交換(初回から考慮!!),抗GBM抗体は病勢を反映する.
\end{itemize}


\subsubsection{サルコイドーシス}

\begin{itemize}
\item 診断は生検によるが,以下の2/5を満たせば臨床的に診断:①BHL,②ACE↑or リゾチーム↑,③sIL-2R↑,④GaシンチまたはFDG-PETでの集積,⑤BALFでリンパ球増多 or CD4/CD8↑.
\item 治療はステロイド or MTX(ステロイドに対して非劣性が示された!NEJM2025),AZA.最重症例はTNF-α阻害薬(インフリキシマブ,アダリムマブ).
\end{itemize}

\subsubsection{その他の間質性肺疾患}

\begin{itemize}
\item リポイド肺炎:脂肪貪食マクロファージ,外因性(ガソリン,灯油,鉱物).4割は無症状,診断はTBLB.対症療法.
\item リポイド肺炎のCTは淡い小葉中心性陰影(acute HPのD/Dに入れる).Crazy-paving appearanceにもなり得る.
\item 好酸球性肺炎のメディエーターとしてeotaxin.
\item AEPの急性期は末梢血は好中球優位,回復期には好酸球>20\%まで上昇する.胸水も出る(片側,両側の両方あり得る)ので心不全と鑑別が必要.
\item AEPは新規の喫煙だけでなく再喫煙,喫煙本数増加などの習慣の変化もリスクとなる.
\item CEPではBALの好酸球比率は>>25\%,40\%以上になることも多い.
\item IgG4-RDの診断基準のカットオフは血清IgG4≧135mg/dL,病理でIgG4/IgG陽性細胞比>40\%,IgG4陽性細胞>10/HPF.
\item LCH: CTで上肺野優位の嚢胞+結節影.CD1a陽性細胞,S-100蛋白陽性.まず(受動喫煙を含めて徹底的に)禁煙,ステロイド+免疫抑制薬,場合により肺移植.※LCHにBRAF-V600E変異陽性例があり,ダブラフェニブ+トラメチニブは投与を考えてもよい.
\item 超合金肺:コバルト,タングステン.
\end{itemize}


\subsubsection{肺癌:分類}

\begin{itemize}
\item 肺腺癌はCK7+,CK20-,TTF1+,NapsinA+.逆にCK7+CK20+は尿路上皮癌,CK7-CK20+は大腸癌.
\item 扁平上皮癌はp40+, p63+, CK5/6+.AdとSqを区別したい場合はp40とTTF-1を見る!
\item 神経内分泌マーカーはクロモグラニンA,シナプトフィジン,CD56.
\item 術前治療の効果判定はRVT(residual viable tumor; 残存生存腫瘍)を用いて行う:Ef0(無効),Ef1a(RVT>2/3),Ef1b(RVT 1/3-2/3),Ef2(RVT<1/3),Ef3(RVT 0).
\item MPR: RVT≦10\%,pCR: RVT 0\%.

\insfig{lk_ef.jpg}{}{原発性肺腫瘍における治療効果の病理学的判定基準}

\end{itemize}


\subsubsection{肺癌:周術期}

\begin{itemize}
\item 縮小手術のエビデンスはGLでは全てIA1-2期に限られている.C/T比≦0.25ならば楔状手術OK,0.25-0.5ならば区域切除,>0.5ならばGLでは区域切除と肺葉切除を並列で推奨(JCOG0802).
\item ただし,JCOG1211(C/T≦0.5,区域切除)ではIA3期まで含めているが,切除マージン≧2cmが求められたことから,GL上はIA3期に対する一律の縮小手術の推奨はない.下図を参照.

\insfig{lk_surgery.png}{0.5}{https://memoinoncology.com/}

\item 周術期のICIはいずれもPD-L1によらずall-comerで使用可(CheckMate77Tは未承認).
\item CheckMate816: 術前はCDDP+PEM+Nivo (Ad), CDDP+GEM+Nivo (Sq), CBDCA+PTX+Nivo (both),3Cやって手術.
\item KEYNOTE-671: 術前はCDDP+PEM+Pembro (Ad), CDDP+GEM+Pembro (Sq),4Cやって手術,術後はPembro (q3w*13C or q6w*7C).
\item AEGEAN: 術前はCDDP/CBDCA+PEM+Durva (Ad), CDDP+GEM+Durva (Sq), CBDCA+PTX+Durva (Sq),4kurやって手術,術後はDurva (q4w*12C).
\item EGFR+の完全切除後はプラチナ併用療法→タグリッソ or 直でタグリッソ.ALK+の完全切除後はアレクチニブonly.
\item 肺尖部胸壁浸潤癌(要はSST)に対する現在の標準治療は術前CRT→手術.ただし術前・術後のDurva追加の単群試験がongoing(JCOG1807C).
\item IMpower010: 術後のプラチナ併用療法→Atezoは,PD-L1≧50\%で「弱く推奨」,PD-L1 1-49\%で「推奨度不明(推奨なし)」.
\end{itemize}


\subsubsection{肺癌:IV期}

\begin{itemize}
\item EGFRのuncommon mutationで多いのはex18 G719X,ex20 S768I,ex21 L861Q.
\item 現行GLでは,EGFR+でタグリッソ単剤でPD後,2nd-lineのABCPは推奨はある(弱い推奨だけど).
\item OncomineだけOK:HER2(エンハーツ,ゾンゲルチニブ)
\item Oncomineだけ不可(Amoy/肺がんCPはOK):ルマケラス
\item AmoyだけOK:新規ROS1(タレトレクチニブ,レポトレクチニブ),新規MET(グマロンチニブ)\footnote{コンパニオン診断は最新情報を確認 https://hokuto.app/post/ZXTQ93WlRz5xYEUeiWoG}
\item 肺がんCPだけ不可(Amoy/OncomineはOK):Ami/Laz,ローブレナ,ロズリートレク,タブレクタ
\item NTRKのロズリートレク,ラロトレクチニブのコンパニオン診断はF1のみ→NTRK陽性例の1st-lineは普通にCx+ICI,2次治療でTRK-TKI使う.
\item 炎症性偽腫瘍でALK陽性.
\end{itemize}


\subsubsection{肺癌:神経内分泌腫瘍}

\begin{itemize}
\item LD-SCLCの標準治療はCCRT→PCI→Durva.ED-SCLCのイムデトラ®は現在は3次治療から.CRS起こったらステロイド+アクテムラ.
\item 気管支・肺カルチノイドは定型・非定型で定義される(病理切片の細胞分裂の個数mitotic rateで分類).

\insfig{carcinoid_classification.jpg}{}{肺癌の病理 ・細胞診断ガイドライン}.

\item カルチノイドの病変が増大傾向,有症状の場合に治療を検討.切除不能の場合は①化学療法(SCLCに準じてPl+ETPを使うが奏効率低い),②ソマトスタチンアナログ(オクトレオチド),③エベロリムス,④ソマトスタチン受容体シンチを行った後でルタテラ®.
\item 胸腺カルチノイドを見たらMEN1を鑑別する.
\item SMARCA4欠損腫瘍:40-50代男性に多い,重喫煙歴あり,未分化腫瘍,IHCで診断,治療はdriver陰性に準じてCx+ICI.
\end{itemize}


\subsubsection{肺癌:支持療法}

\begin{itemize}
\item CDDPは制吐薬を4剤併用:5-HT3(要はセロトニン)受容体拮抗薬(第2世代:パロノセトロン アロキシ®),DEX,NK1受容体拮抗薬(イメンド,プロイメンド,アロカリス),オランザピン.
\item 中等症催吐性リスク薬剤のうち,CBDCAのみ「5-HT3+DEX+NK1」(オランザピン以外の3剤併用).それ以外の薬剤は「5-HT3+DEX」.
\item G-CSFの一次予防はFN発症率>20\% のレジメンに適応(DTX+RAMなど).
\item G-CSFの治療的投与は非推奨,FN後のG-CSFの二次予防的な投与はやっていいが,化学療法の減量,延期が基本である
\end{itemize}

\subsubsection{胸腺腫,胸腺癌}

\begin{itemize}
\item 胸腺腫の手術合併症は横隔神経麻痺,反回神経麻痺,※術後重症筋無力症(MG合併がない無症候性の胸腺腫であっても術後に1-3\% で発症する!)
\item 胸腺腫+低γグロブリン→Good症候群で予後不良.
\item 胸腺腫の正岡分類:IIB期で肉眼的浸潤,IVA期で胸膜・心膜播種,IVB期でリンパ行性,血行性転移.

\insdualfig{thymoma_masaoka.jpg}{1}{thymoma_who.jpg}{1}{朝倉内科学(第12版)}

\item 切除可能な胸腺腫・胸腺癌に針生検は禁忌.
\item 胸腺腫はIII期まで1st choiceは外科的切除.不完全切除例はPORTを行う(ただしIII期胸腺腫は完全切除でもPORTやってもいい).
\item 胸腺癌もIII期まで1st choiceは外科的切除.II-III期は完全切除でもPORT必須.
\item 切除不能胸腺腫は化学療法(ドキソルビシン,シスプラチン,CY±ビンクリスチン).ICIは治療関連死の可能性あり不可!
\item 切除不能胸腺癌は化学療法(CBDCA+PTX or CBDCA+AMR).2nd-lineはレンバチニブ,TMB高ければ(癌なら)Pembro使っても良い.
\item ※切除不能胸腺癌の1st-lineでCBDCA+PTX+Atezoの有効性を確認(MARBLE試験).
\item セミノーマはhCG上昇(10-20\%),非セミノーマはAFP上昇(50-70\%).
\end{itemize}


\subsubsection{MPM}

\begin{itemize}
\item 悪性中皮腫は胸膜原発が80-85\%,腹膜原発が10-15\%,中皮腫の70\%で石綿曝露あり.
\item MPMの診断はまずIHCで癌腫の除外を行う:カルレチニン, WT-1, D2-40, CD5/6陽性.逆に腺癌マーカー(CEA,TTF-1,Claudin-4,Napsin-A)は陰性.
\item 癌腫が除外できたら,MPMと反応性中皮過形成の鑑別.組織学的に脂肪組織以深への浸潤の有無を認めればMPMと診断できるが,細胞診(セルブロック)では不可.
\item 一般的には反応性中皮はデスミンが陽性,上皮型MPMはデスミンが陰性になるが,例外も多く確定診断にはならない.
\item BAP1欠失,MTAP欠失,CDKN2A遺伝子のホモ欠失があれば,反応性中皮ではなくMPMと診断しうる.→BAP1とMTAPはIHC,CDKN2AはFISHで判断.
\item 脂肪組織への浸潤はないが,反応性中皮ではなく絶対MPMというケースがある:前浸潤性中皮腫(mesothelioma in situ)→やはりCDKN2Aのlossがあれば診断可.
\item 肉腫型MPMと肺肉腫様癌の100\%の鑑別は難しいので,CT所見を総合的に判断する.
\item 肉腫型MPM:サイトケラチン陽性,GATA3びまん性に陽性.肉腫様癌:MUC4陽性.※MTAPとBAP1は肉腫型では使えない!
\item 中皮腫細胞は電顕で微絨毛の発達がみられる(ただし全例で確認できるわけではない).ヒアルロニダーゼ消化試験陽性.
\item MPMの外科治療は上皮型(または二相性)MPMのみ適応がある.壁側胸膜と肺表面の臓側胸膜のみを切除:胸膜切除/肺剝皮術(P/D).
\item MPMに外科治療を行う場合,術前にCDDP+PEMは必須.EPP(胸膜肺全摘)ならば術後に片側胸郭RT.P/Dの場合はRTはせず,CDDP+PEM追加.
\item ※1st-line: Nivo+Ipi or CBDCA+PEM+Pembro(KEYNOTE-483試験の結果から2025/5 承認!)
\item 2nd-line: PEM, CDDP+PEM(ただしPEM維持療法はPFS延長しない.維持療法をGEMにすれば有効性あり),VNR+GEMなどを適宜検討する.
\end{itemize}


\subsubsection{じん肺の管理}

\begin{itemize}
\item じん肺の管理は管理1-4に分類され,管理2以上のものは健康管理手帳,公費で健診受けられる.
\item じん肺の診断時にDrが行う検査は①胸部Xp ②PFT → 労働局へ提出.
\item \textbf{じん肺一般の}労災基準:管理4全員,管理2/3でTb/Tb胸膜炎,続発性気胸/気管支炎,気管支拡張症, 肺癌.
\item \textbf{石綿の}労災基準:じん肺の労災基準を満たす場合(じん肺ではなく特に石綿肺と呼ぶ),中皮腫,良性石綿胸水,びまん性胸膜肥厚,肺癌(石綿小体の数が重要なことがありBALはできるだけ行う)
\item 石綿の高濃度曝露→良性石綿胸水,びまん性胸膜肥厚.低濃度曝露→胸膜プラーク,MPM.
\item 良性石綿胸水の定義:数ヶ月で自然軽快,胸水確認後3年は悪性腫瘍がない(両側に生じてもOK).
\item 石綿関連疾患で職業曝露以外だと労災は受けられないが,石綿健康被害救済制度の対象となる.MPM,石綿肺癌,良性石綿胸水,びまん性胸膜肥厚いずれも適応.
\end{itemize}


\subsubsection{PH}

\begin{itemize}
\item 肺の有効血管床の50-60\%が障害されるとPA圧上昇.
\item PHはPO2の値に関係なくHOT保険適応.
\item PAHの定義(2022年,ESC/ERSガイドライン)は「mPAP≧20mmHg」,ただし難病の基準は「≧25mmHg」.
\item PAHを見たらまずHIV測定,AUS(門脈圧亢進の除外).
\item 薬剤性PHの原因:食欲抑制薬(アミノレックス,フェンフルラミン),ダサチニブ.
\item RHCによるPAH診断:mPAP≧25(本邦のGLが更新されていないため),PAWP≦15mmHg,PVR>3Wood.
\item RHC中に行う肺血管反応性試験:NO吸入,エポプロステノール持続静注,アデノシン静注,Ca blocker内服.
\item WHO-PH機能分類:II度(普通の労作時に症状),III度(軽労作でも症状),IV度(常時症状).まずはII度を目指す.
\item PHリスク評価の項目:①疾患進行,②失神,③WHO機能分類,④6MWT,⑤CPET,⑥BNP/NT-proBNP,⑦TTE/MRI所見,⑧カテーテル所見.

\insfig{pa_risk.jpg}{}{https://www.msdconnect.jp/therapeutic-areas/ph/pah/pah-risk-assessment/}

\item 呼吸器疾患(COPD, IP)によるPHには吸入トレプロスチニル(トレプロスト®)のみ推奨!その他の血管拡張薬(ERA, PDE5i)などは全て非推奨.
\item SSc-PAHにはiPAHと同様に血管拡張薬を1stで使う,その他のCTD-PAHにはまず原疾患の治療(ステロイド,免疫抑制薬).
\item PAHの治療オプションはERA阻害薬,PDE5i(→NO増やす),sGC刺激薬,PGI2誘導体,PGI2受容体作動薬(セレキシパグ).
\item PAHの治療の基本はERA+PDE5i→ウプトラビ追加→PGI2追加(吸入,皮下注,静注).
\item ERAはボセンタン(トラクリア®:肝障害リスク高い),アンブリセンタン(ヴォリブリス®:IP増悪リスク),マシテンタン(オプスミット®).
\item PDE5iが効果不十分の場合はPDE5iに変えてsGC(アデムパス®)使うが,sBP<95mmHgでは増量できない.
\item PAHの新薬としてアクチビンシグナル伝達阻害薬(ソタテルセプト).
\end{itemize}

\subsubsection{CTEPH}

\begin{itemize}
\item CTEPH:適切な抗凝固薬治療を3-6ヶ月行ってもPHが継続している場合,肺換気血流シンチが診断に必須.70歳代がピーク.
\item 日本ではCTEPHの原疾患はバラバラ:PE既往例,DVT合併例はともに30\%.血液凝固異常が15\%,心疾患と悪性腫瘍が10\%くらい.
\item CTEPHのリスク:PEの病歴,血管内デバイス,炎症性腸疾患,血小板血症/多血症,脾摘術後,APS,高用量甲状腺ホルモン補充.
\item CTEPHでDVTが原因の場合,IVCフィルターは永久型を使用する.
\item 治療はまず全例でワルファリン or DOACsで,終生投与
\item CTEPHに対する外科治療は,operableなら(病変に到達可能なら)肺動脈血栓内膜摘除術 (PEA).PEAの適応がなければバルーン肺動脈形成術 (BPA) and/or 肺血管拡張薬(アデムパス®,ウプトラビ®が保険適応).PEA後にPHが残存する場合もBPA,肺血管拡張薬いずれもやっていい.

\end{itemize}


\subsubsection{PE}

\begin{itemize}
\item ECGのS1Q3T3: I誘導で深いS波,III誘導でQ波と陰性T波(ただしPEの20\%).
\item 急性PEで血行動態が安定している場合はイグザレルト or エリキュースをローディングドーズで開始でもよい(初手リクシアナはダメ).
\item ショックや低血圧が遷延するPEにはt-PAによる血栓溶解を行ってもよい(カテーテル治療は専門施設でない限りはしない).
\item cancer VTEに対しては国際標準は低分子ヘパリンだが,日本では未承認.GLでもDOACsがclass IA推奨.
\end{itemize}


\subsubsection{血液疾患に伴う肺病変}

\begin{itemize}
\item 肺原発の悪性リンパ腫は圧倒的多数がMALTリンパ腫(70-90\%),次にDLBCL.性差はなし.
\item 膿胸関連リンパ腫(PAL):慢性膿胸,結核性胸膜炎への人工気胸術後>20年,胸腔内に発生するDLBCL.
\item 原発性滲出性リンパ腫(PEL):HIVや移植と関連する,HHV-8陽性,胸水など体液貯留のみ.超予後不良(1y-OS 30\%).
\item HTLV-1関連肺病変(HTLV-1 associated bronchiolo-alveolar disorder: HABA):リンパ路障害(小葉間隔壁の肥厚)and/or DPBに類似したtree-in-bud appearance.

\end{itemize}


\subsubsection{形態異常}

\begin{itemize}
\item 肺動静脈瘻のほとんどは遺伝性出血性毛細血管拡張症(HHT:要はOsler病)に由来する:常染色体優性遺伝.
\item HHT: ENG,ALK-1,SMAD4.粘膜や皮膚の毛細血管拡張病変.BrMRI(造影),CECT(肺+肝臓),コントラスト心エコー,AUS,脊髄MRI,GS/CS.
\item 肺動静脈瘻の症状:古典的にはチアノーゼ(R-L shuntによるde-sat),多血症,ばち指.ダイビングNG,菌血症リスクあるので歯科治療のABxは必須.門脈圧亢進症や肝動静脈瘻によるPHも起こりえる.
\item 肺分画症でみられる肺組織は気管支との交通がないので,ガス交換ができない.
\item 肺葉内肺分画症(正常肺葉内に存在,肺静脈に還流,気道感染を反復),肺葉外肺分画症(固有の臓側胸膜を持つ,奇静脈や門脈に還流,合併奇形あり,無症状).
\item 正常な肺組織に大動脈から異常血管が分岐して流入する(気管支と交通してるのであくまでガス交換はできる)=「肺底区動脈大動脈起始症 anomalous systemic arterializations of lung without sequestration」
\end{itemize}


\subsubsection{稀な疾患群}

\begin{itemize}
\item Platypnea-orthodeoxia症候群:心内シャント(卵円孔開存),V-Q mismatch(肝肺症候群,IP).
\item 肝肺症候群:慢性肝疾患に低酸素血症を伴う病態で,肺の末梢血管拡張→肺胞が遠くなるのでde-saturation→つまり肺内シャント.コントラスト心エコーで診断.
\item 黄色爪症候群(Yellow nail syndrome, YNS):黄色爪,四肢(下腿)リンパ浮腫.呼吸器症状の合併が多い(BE, 慢性気管支炎).ブシラミン,金,D-ペニシラミンと関連.
\item YNSの胸水:リンパ球優位,TG上昇.
\item YNSの治療は対症療法のみ:MLs, ビタミンB3/E, 亜鉛.
\item MCDはVEGF,AAアミロイドーシス合併しうる,HHV-8が関連する病型は予後不良.
\item 肺胞微石症:SLC34A2(リンの運搬蛋白),常染色体劣性遺伝,リン酸カルシウムを主成分とする微小結石が蓄積する.胸部X線でびまん性の微小粒状影(snow storm appearance).石は<1mmでCTの分解能を下回るためすりガラス影として観察される.
\item Birt-Hogg-Dubé症候群:FLCN遺伝子の異常,常染色体優性遺伝.皮膚・肺・腎.皮膚(顔面,頸部,上半身に繊維毛包腫),肺(肺嚢胞は下肺野縦隔側に好発),腎(嫌色素性腫瘍またはオンコサイトーマ).
\item 特発性樹枝状肺骨化症:多数の微小な骨組織が肺組織に形成される病態.IPFでHRCTでびまん性肺骨化がしばしば検出される(肺胞微石症とは別).
\item CCHS(先天性中枢性低換気症候群):PHOX2b遺伝子異常が原因,O2↓やCO2↑に対する換気応答がなくなり,non-REM睡眠時にPaCO2が著増.炭酸ガス換気応答テスト (VR-CO2),NPPV.
\item Swyer-James症候群:左肺におきやすい.CXpで一側肺の透過性亢進と肺血管陰影の減少をきたす.気管支造影亜区域支以遠の造影不良所見,肺血流シンチでは片側性の完全血流欠損を示す.無症状では経過観察のみ.
\end{itemize}


\subsubsection{胸郭疾患}

\begin{itemize}
\item 乳び胸:胸管の破綻が第5胸椎より上で生じると左胸水,それより下だと右胸水になる.
\item 乳び胸の治療はオクトレオチド.排液量>1000mLであれば胸管塞栓術の適応はある.
\item 横隔神経麻痺:第3-5頚椎から出る.診断は臥位Xp.
\item 横隔膜弛緩症:先天性は胎生期の形成不全.後天性は横隔膜の鈍的障害.左の挙上が多い.
\item 漏斗胸の手術適応の判断はHaller index:(胸腔横径)÷(胸骨後面と椎骨前面の距離)>3.25.
\item 大量血胸の開胸止血の適応:①胸腔ドレナージ開始時に1000mL,②開始1時間で1500mL,③開始2-4時間で>200mL/h,④持続的な輸血が必要.ただし鈍的外傷は開胸不要なことも多い.
\end{itemize}


\subsubsection{禁煙外来}

\begin{itemize}
\item 禁煙治療の条件:①禁煙意思あり,②ニコチン依存症スクリーニング>5点,③BI>200(35歳未満は制限なし),④文書による同意(+過去1年以内の禁煙治療歴なし)
\item バレニクリン(チャンピックス®)は漸増して12週使用.最初の1週間だけ喫煙OK!ドライバーには処方不可.
\item ニコチンパッチ(ニコチネル®)は漸減して8週使用.
\item 日本の禁煙治療の成功率は20\%.
\end{itemize}


\subsubsection{造血幹細胞移植の呼吸器合併症}

\begin{itemize}
\item 感染症:<30日でまず普通の細菌,カンジダ,\textbf{ヘルペス}.30-100日で\textbf{CMV},アデノ.>100日で\textbf{帯状疱疹},莢膜を持つ細菌による慢性感染症(ヘルペス→サイトメガロ→VZVの順).
\insfig{transplant_infection.png}{0.5}{https://www.jstct.or.jp/}

\item 生着症候群:造血幹細胞が骨髄で実際に造血を始めるときにサイトカインが放出され急性GVHDに類似した症状が起こる.肺病変はPeri-engraftment Respiratory Distress Syndrome (PERDS) と呼びARDSに類似した病態.ステロイド.
\item 生着〜移植後100日(移植後早期):びまん性肺胞出血 (DAH:死亡率60-100\%で超予後不良),特発性肺炎症候群 (IPS:ARDSに類似した病態,ステロイドへの反応は鈍い).
\item 移植後>100日(移植後晩期):BOS,PPFE.

\end{itemize}

\subsubsection{肺移植}

\begin{itemize}
\item レシピエントの年齢は両肺移植<55歳,片肺移植<60歳.
\item レシピエントがHIV抗体陽性でも移植可能.ひどい慢性気道感染があっても(一応)移植可能.
\item 除外条件は高度胸郭変形,T-Bil>2.5, Cre>1.5,肺外の活動性感染巣,悪性疾患,薬物依存,極端な肥満,現喫煙.
\item 肺移植後の移植片慢性機能不全(CLAD)も肺移植の適応となる.CLADの病型はBO(末梢気道の炎症,線維化と狭窄)またはRAS(Restrictive allograft syndrome:拘束性変化).
\end{itemize}
