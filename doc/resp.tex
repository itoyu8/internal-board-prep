%!TEX root = ../board-prep.parent.tex


\section{呼吸器}

\subsection{総論}

\subsubsection{呼吸器の発生と解剖}

\begin{itemize}
\item 前腸(foregut)の腹側に肺芽が出現 → 分岐と伸長を繰り返して肺胞.
\item 胚芽期(26日-6w:気管支まで)→偽腺管期(6-16w:終末細気管支まで)→細管期(16-28w:呼吸細気管支まで,サーファクタント産生開始)→嚢状期(28-36w:間質が減少,サーファクタント分泌完成,胎外生活可能)→肺胞期(36w-).
\item 細気管支(φ2mm)の特徴:気管軟骨と気管支腺が消失,club cell>線毛上皮細胞,Millerの二次小葉を支配.
\item Club cell(旧Clala cell)は分裂能があり,CCSP (club cell secretory protein)を分泌する.
\item 細気管支=二次小葉,終末細気管支=細葉(細葉が集まって二次小葉を形成)
\item 呼吸細気管支の定義:壁に肺胞が付着した細気管支.
\item 終末〜呼吸細気管支から,中枢側に向かって逆行する反回枝(娘枝)が出る.
\item 肺胞の直径は0.1-0.2mm.
\item 気管支動脈は,右は肋間動脈,左は胸部大動脈から分岐.
\item 気管支動脈の血流量は心拍出量の1\%.
\item 胸膜中皮細胞は中胚葉由来,水代謝に関与する.
\item 放射線感受性が高い細胞:2型肺胞上皮,血管内皮.
\end{itemize}


\subsubsection{呼吸生理}

\begin{itemize}
\item 化学受容器は中枢:延髄腹側(PaCO2),末梢:頸動脈小体,大動脈小体など複数.
\item 低酸素換気抑制:低酸素状態が長く続くと換気量↑のレスポンスが鈍くなる.
\item PaCO2↑に対応する換気量upはPaO2↑で鈍る(ただのCO2ナルコーシス),睡眠でも鈍る.
\item 酸素解離曲線の右方シフトは体温↑,アシデミア,2,3-DPG↑.
\item 血管内皮細胞はACEを分泌,ブラジキニンとセロトニンを不活化(分解).
\item アラキドン酸カスケードの起点はcPLA2 (cytosolic phospholipase A2).COX系の脂質メディエーターはPGとTXA2,5-LO系の脂質メディエーターはLT.
\end{itemize}


\subsubsection{疫学}

\begin{itemize}
\item 新規の肺癌:男性8万人,女性4万人.
\item 塗抹陽性結核:4例/10万人.
\item 結核死:2000人/年.90歳以上の結核患者の死亡率:50\%.LTBI患者のうち医療従事者:25\%.
\item 喘息死:2000人/年(減少),COPD死:18000人/年.
\item 小児喘息のうち成人喘息への移行:30\%.
\end{itemize}


\subsubsection{主要徴候と身体所見}

\begin{itemize}
\item Miller \& Jones分類は肉眼所見,P2以上でgood quality.
\item Geckler分類は顕微鏡所見,4/5群(吸引検体なら6群)でgood quality.
\item ACT:\textbf{20-24点}でコントロール不十分,\textbf{<20点}でコントロール不良.MCIDは3点.小児にはC-ACT.
\item ACTの項目:日常生活への支障,息切れ,夜間の中途覚醒,SABA使用回数,自身での喘息コントロールの自覚.
\item 修正Borgスケール:0〜10,0.5 =「非常に弱い息切れ」.
\item mMRCグレード3:「100mまたは数分歩いて息切れ」.

\insdualfig{dyspnea_fhj.jpg}{1}{dyspnea_mmrc.jpg}{1}{倉原先生のブログより}

\item 嗄声をきたす癌:甲状腺癌>肺癌>食道癌.
\item ばち指:DPD/IPD>1.0(爪甲基部の厚みの方がDIP関節の厚みよりも分厚い = これがばち指の特徴).シャムロス徴候+.指末端でPDGFやVEGFが分泌.
\item 肺性肥大性骨関節症はAd(やSq)に合併.ばち指,四肢長管骨の骨膜新生,関節炎.
\item 抗VGKC抗体:SCLC,胸腺腫.抗VGCC抗体:Lambert-Eaton症候群(つまりSCLC).
\item MGで抗MuSK抗体陽性なら胸腺切除術は非推奨(抗AChR抗体陽性ならDo).
\item Hoover徴候:COPD,振子呼吸:肺結核後遺症.
\end{itemize}


\subsubsection{検査}

\begin{itemize}
\item プリックテストの判定は15分後.H1bは検査4-5日前,H2bは24時間前に中止.LTRAは中止不要.
\item 皮内テストはプリックテスト陰性例に行う.0.02mLを前腕屈側に注射.15分後に判定.
\item 喀痰細胞診は常温で12時間以内,冷蔵で24時間以内に検査する.
\item 被曝量:CXp1回で0.04mSv, 胸部CT1回で7.8mSv.
\item 滲出性胸水の補助診断:①胸水T-Chol>55mg/dL,②胸水T-Chol/血清T-Chol>0.3,③血清Alb-胸水Alb<1.2g/dL.
\item 胸水ADA上昇:Ly優位→結核,リウマチ性胸水.Neu優位→膿胸,肺炎随伴性胸水.他にリンパ腫やIgG4RDでもADA上昇.
\item 良性石綿胸水の胸水はEosino優位.MPMの胸水はヒアルロン酸>10万ng/mL,SMRP上昇(>8-15nmol/L).
\end{itemize}


\subsubsection{PFT}

\begin{itemize}
\item 肺癌手術の目安:術前\%FEV1>50\%,術前\%DLco>50\%.術後予測1秒量>800mL,術後予測\%1秒量>30\%.
\item DLco測定時の混合ガス:N2, O2, He, CO(COは拡散+希釈,Heは希釈をみる).
\item DLcoの注意点:①smokerは低めに出る(検査前24時間は喫煙),②食後2時間,運動直後を避ける.
\item DLco/VA:ガス交換面積あたりのDLcoを見る → COPD(気腫型)ではDLcoより低めに出る,IPではDLcoより高めに出る.
\item アストグラフ:BAの気道過敏性試験.メサコリン吸入.
\item V50/V25:それぞれ肺気量の50\%(25\%)のときの呼気速度.>3.5で末梢気道閉塞.
\item CV (closing volume):末梢気道閉塞を見る.100\% O2を吸うと肺底部に溜まって肺尖部は空気(N2)が相対的に多い→息を吐くとまず下肺野のO2が先に出る→さらに息を吐いてN2が増え始めたところで下肺野は空気が抜けて(or 末梢気道病変のために)つぶれたと考える.\item CVは普通10\%,20-25\%を超えると末梢気道閉塞.
\item 炭酸ガス換気応答テスト (VR-CO2):呼気を繰り返し吸わせてCO2↑→普通は換気量up.先天性中枢性低換気症候群 (CCHS) では換気量upの反応がなくなる.
\item 広域周波オシレーション法:音響スピーカーなどによる工学的な空気振動を安静呼吸で吸ってもらいPCで解析.呼吸インピーダンス (respiratory system impedance; Zrs) を測定し,これを呼吸抵抗 (respiratory system resistance; Rrs),呼吸リアクタンス (respiratory system reactance; Xrs)に分解する.
\item BA, COPDではRrsが増加し,特に低周波数領域でRrsが高くなる.
\end{itemize}


\subsubsection{BFS}

\begin{itemize}
\item BALの正常所見:細胞数〜13万,Mph 90\%,Ly 10-15\%,CD4/8 1-2.
\item BALのCD4/8比↑:サルコイドーシス,結核,ベリリウム肺,慢性HP.CD4/8比↓:CTD-ILD(NSIP, COP),急性〜亜急性HP,珪肺??
\item 100\% O2投与下で使用可能なのはマイクロ波凝固療法とクライオ.APCは気道穿孔リスク低い.
\item BTが作用するのは平滑筋細胞(減らす),線維芽細胞(リモデリング改善),迷走神経.
\end{itemize}


\subsubsection{医療費制度}

\begin{itemize}
\item 指定難病による医療費助成:指定難病のうち重症度が一定以上の場合,自己負担が基本2割となり,さらに1ヶ月あたりの負担上限額が設定される.
\item ただし,指定難病で医療費助成を受けられない「軽症者」であっても,月ごとの医療費総額(10割負担額)が33,330円を超える月が、年3回以上ある場合は医療費助成の対象となる(いわゆる軽症高額).
\item 身体障害者手帳:障害者控除(税金),交通機関の割引運賃などの福祉サービス.心身障害者医療費助成としてさらに医療費の負担が減る可能性も.
\item 身体障害者手帳(呼吸器機能障害)の判断項目は,①指数(予測肺活量1秒率:つまり1秒量/予測肺活量,FEV1でも\%FEV1でもない!!!)②PaO2.
\item 障害者手帳1級:指数≦20 or PaO2≦50Torr,3級:指数20-30 or PaO2 50-60,4級:指数30-40 or PaO2 60-70.
\item 「お題目」としては,手帳1級がmMRC 4,3級がmMRC 3,4級がmMRC 1-2.

\end{itemize}

\subsection{疾患}

\subsubsection{SAS}

\begin{itemize}
\item エプワース眠気尺度:>11点で高リスク → 簡易PSG(アプノモニター)に進む.

\insfig{epworth_scale}{}{https://hirai-naika-geka.com}

\item 簡易PSGでは純睡眠時間は分からないので,代わりにモニター装着時間を使う → AHIではなくREI (respiratory event index).普通にAHIより低く出る.
\item PSGの記録チャンネル7個:脳波,眼電図,筋電図,心電図,気流,呼吸努力,SpO2.
\item OSA重症度(AHI):軽症5-15,中等症15-30,重症>30.
\item CPAP保険適応基準:アプノモニターでAHI>40,PSGでAHI>20.
\end{itemize}


\subsubsection{細菌感染症}

\begin{itemize}
\item A-DROPはSpO2<90\% (r/a),I-ROADはSpO2<90\% (FiO2 35\%).
\item CURB65:年齢>65,呼吸数>30(SpO2ではない).残りはA-DROPと同じ.
\item NHCAP:①施設,②<90日以内に退院,③要介護,④定期的な血管内治療(透析,ケモ,免疫抑制薬,ABx).
\item NHCAPもA-DROP使う.
\item 耐性菌リスク:①<90日の静注ABx投与歴,②<90日の2日以上の入院歴,③免疫抑制薬,④活動性低下(PS≥3,BI<50,歩行不能,経管栄養,CV)

\insdualfig{adrop.jpg}{1}{iroad.jpg}{0.9}{https://knowledge.nurse-senka.jp}

\item 定型肺炎と非定型肺炎の鑑別(諸外国には浸透していない):①<60歳,②基礎疾患なし,③ひどい咳,④聴診でcrackleなし,⑤喀痰から起因菌生えない,⑥WBC<10000.3/5または4/6以上で非定型疑い.
\item 多剤耐性緑膿菌(MDRP)の定義:カルバペネム,アミノグリコシド,キノロンに耐性.
\item 耐性グラム陰性桿菌(CREなど)の治療薬:コリスチン,チゲサイクリン,セフィデロコル(フェトロージャ®),セフトロザン・タゾバクタム(ザバクサ®),イミペネム・シラスタチン・レレバクタム(レカルブリオ®)
など.
\item 肺炎球菌ワクチン:経過措置は終了し,>65歳 or 60-65歳の重症患者にPPSV23を1回だけ定期接種する.PPSV23の代わりにPCV20,PCV21,PCV15→23の連続接種はありだが,いずれもoption.

\insfig{pneumonitis_vaccine.jpg}{}{65歳以上の成人に対する肺炎球菌ワクチン接種に関する考え方(第7版)}

\item 誤嚥性肺炎に対する嚥下機能評価:①反復唾液嚥下テスト(RSST:30秒で唾液3回以上飲めればOK),②簡易嚥下誘発試験(SSPT:中咽頭までカテーテル入れて0.4mLの5\% Gluを注入,3秒以内に嚥下すればOK),③水飲み試験(WST:普通に水飲んでもらう).
\end{itemize}


\subsubsection{百日咳}

\begin{itemize}
\item 血清抗体は発症2週〜.PCR/LAMPは〜発症3週で採取する.
\item PT-IgGはワクチンの影響を受けるので基本的に単血清では診断できない.ただし>100 EU/mLなら新規感染.
\item PT-IgGでペア血清をとる場合,「1回目<10 EU/mL,2回目>10 EU/mL」or「1回目 10-100 EU/mL,2回目で1回目の2倍以上」ならOK.
\item IgM/IgAはワクチンの影響を受けないが陽性率低い.IgMは発症2週,IgAは発症3週でピーク.
\end{itemize}


\subsubsection{非定型肺炎}

\begin{itemize}
\item マイコプラズマの診断は抗原,LAMP法,抗体(PA法ただしペア血清).培養するならPPLO培地.
\item PA法は主にIgM(と少しIgG)を測る.day7から上昇.MLs無効ならTC or NQ.
\item クラミジア肺炎は潜伏期間3-4週,5類感染症(定点把握),IgM(>10日)やIgGのペア血清で診断する.
\item レジオネラの観察にはGiemenez染色,アクリジンオレンジ染色.培養はBCYE-α培地,WYO培地を使う.
\end{itemize}


\subsubsection{結核}

\begin{itemize}
\item T-SPOTの偽陽性はM.kansasii(有名), M.szulgai, M.marinum, M.gordonae.
\item ガフキーは1〜10号で判定.1+はG2(数視野に1個),2+はG5(1視野に4-6個),3+はG9(1視野に51-100個).
\item 培養は液体培地(MGIT法)で6週間,固形培地(小川培地)で8週間.INH耐性は液体培地だと偽陽性になりやすい(固形培地までやるとSにひっくり返ることがある).
\item 菌の同定は,培養前の検体(喀痰など)に対してPCR(TB,avium, intracellulareが検出可能),培養陽性になった菌株に対して質量分析法(159菌種).DDHマイコバクテリア法は受託中止(小川培地から生えた菌しか検査できず,しかも18種類しか検査できなかった).
\item A法:2HREZ+4HR,B法:2HRE+7HR.
\item A法,B法で+3HRを追加する条件は,①治療開始2ヶ月以降で培養陽性,②空洞形成など重症結核,③再治療,④免疫低下.
\item 腎障害がある場合はEBとPZAを週3回に減量(実際にはB法でHR連日+EB週3回にする).
\item RFPとINHは減感作療法が可能.いったん両方休薬して1剤ずつ減感作プロトコルに従ってdose upする.
\item RFP+VRCZ併用禁忌.INHはヒスチジンが蓄積して発疹出やすい.
\item PZAの禁忌は妊婦,(ADL不良)高齢者.PZA眼障害のリスクは腎不全,DM,肝障害,低栄養,喫煙,飲酒.
\end{itemize}


\subsubsection{耐性結核}

\begin{itemize}
\item INH耐性結核にはINHの代わりにLVFX.GLの推奨は6RELZ (RFP+EB+LVFX+PZA)で,軽症例に限っては2RELZ+4RELも許容範囲.PZAが一切使えない場合は12REL.
\item RFP耐性結核にはRFPの代わりにLVFX.推奨は①6HELZ+12HEL,②6HESL (INH+EB+SM+LVFX)+12HEL.菌陰性化から18ヶ月.
\item 多剤耐性結核の場合,①キードラッグはLVFXとBDQ,②なるべくLZD,③EB, PZA, DLM(デラマニド), CFZ(クロファジミン), CS(サイクロセリン)の中から選んで,合計5剤使う.基本は菌陰性化から18ヶ月.
\item 超多剤耐性結核→NQ耐性かつKM,AMK,カプレオマイシンのいずれかに耐性.
\end{itemize}


\subsubsection{LTBI}

\begin{itemize}
\item LTBIの治療レジメン:4HR or 6H,INHが使用できない場合に限り4R(RFP投与後のRFP耐性結核の発症リスクが上昇するか不明のため).
\item LTBIの治療適応は下記を参照だが,①接触者検診で陽性は治療,②DMはコントロール不良の場合のみ治療,③PSL>15mg, >1ヶ月は治療.
\item RAはデフォルトでLTBIの高リスクで,TNF-α阻害薬でさらにリスク上昇.TNF-α阻害薬を使う場合,3週前からLTBI治療スタート.

\insfig{ltbi.jpg}{}{潜在性結核感染症治療指針}

\end{itemize}


\subsubsection{NTM}

\begin{itemize}
\item 迅速発育菌はabscessus, Chelonae, Fortuitum.
\item NTMの診断基準の基本は臨床診断(CT所見)+細菌学的診断(喀痰2回培養陽性,BAL/気管支洗浄液で1回陽性,喀痰1回陽性+肺病理がcompatible).
\item 2024年のNTM暫定的診断基準では,細菌学的診断として(喀痰2回陽性の代わりに)「喀痰1回陽性+キャピリアMAC抗体陽性」or「喀痰1回陽性+胃液1回陽性」でもOK.
\item MAC症に対する治療は,CAMの代わりにAZM (250mg/d)使用可.FC型/重症NB型では開始3-6ヶ月はSM/AMKを併用.
\item RECAMを6ヶ月継続しても治癒しない(←厳密な定義はない)難治例はRECAMを継続しつつSM/AMK/ALIS.
\item M.kansasii → INH+RFP+AZM or INH+RFP+EB(T-SPOT偽陽性のため結核のレジメンと近い).
\item M.abscessus → IMP/CS+AMK+クロファジミン(ランプレン®).が基本.MLs感受性ならAZMも追加する.
\item M.absesscusの中でmassilienseはerm41がないのでMLs誘導耐性を起こさない.ただしabscessusの中でerm41を有するが,特にC28 sequevarである場合,例外的にMLs誘導耐性を起こさずMLs使える.
\end{itemize}


\subsubsection{その他の細菌}

\begin{itemize}
\item Actinomyces: GPR,胸壁に病変が進展して瘻孔形成しうる.1st choiceはペニシリン系.
\item ノカルジア:GPR,Ziehl-Neelsen染色で染色される.脳膿瘍リスク.1st choiceはTMP-SMX.
\item 肺吸虫症は日本では宮崎肺吸虫症とWesterman肺吸虫症がツートップ.気胸,浸潤影,結節,空洞影,胸水など.プラジカンデル.
\item トキソカラ症は牛,鶏の生食で感染.アルベンダゾール or メベンダゾール.
\item 糞線虫症は沖縄,奄美に多い.ARDS合併する,イベルメクチン.
\end{itemize}


\subsubsection{PCP}

\begin{itemize}
\item PCPは栄養体であればGiemsa染色,Diff-Quik法で染色可能だが,嚢胞体は染まらない.農法対を染めるにはメテナミン銀染色,トルイジンブルーO染色が必要.
\item HIV-PCPでは嚢胞きたしうる.
\item ペンタミジン(ベナンバックス)を使う場合,PCP治療では静注のみ,予防では静注 or 吸入(いずれも月1回程度,吸入は効果落ちる).
\end{itemize}


\subsubsection{肺真菌症}

\begin{itemize}
\item FLCZはカンジダ+クリプト(しかもC.kruseiとC.glabrataには無効→これらにはMCFG使う).ITCZはアスペルギルス有効だがbioavailability超低い.
\item VRCZはTDM必要,副作用多い→眼,肝,皮膚(光線過敏,紅斑).
\item ISAVはCPA+IPA両方の適応あり.サイクロデキストリン入っていないので腎障害あってもdiv可能.
\item アスペルギルスのうちterreusとflavusはAMP-Bが効きにくい.
\item ABPM診断基準.現在はアスペルギルス沈降抗体は受託中止,アスペルギルスIgG抗体を使用する.

\insfig{abpm.png}{}{https://pulmonary.exblog.jp/30457004/}

\item IPAのCTでは病初期ではhalo sign,回復期ではair crescent sign.
\item 肺クリプトコッカス症は届出不要,播種性クリプトコッカス症は5類感染症.Cryptococcus neoformansがほとんどだがC.gattiの報告例もある.
\item 肺クリプトコッカスの治療はFLCZ,播種病変がある場合はAMP-B+5-FCを最低2週(L-AMBのエビデンスは少ない),その後FLCZによる地固め.
\item ムーコル症はAMP-B, L-AMB, PSCZなどを使用する.可能であれば外科治療も併用.
\end{itemize}


\subsubsection{BA}

\begin{itemize}
\item LTC4, TGF-βで気道平滑筋が肥厚する.PGE2は例外的に気道拡張.
\item Th2細胞→IL-4(B細胞をクラススイッチさせてIgE産生,気道粘液),IL-5(好酸球増殖),IL-13(気道過敏性,IgE産生). 
\item 好酸球からMBP (Major basic protein)分泌→気道過敏性亢進.
\item IL-25, IL-33,TSLP → ILC2(細胞表面にIL-4Rが発現,ステロイド抵抗性) → IL-5, IL-13.
\item 粘液細胞からMUC5B, 杯細胞からMUC5ACが分泌されて気道上皮の表層(ゲル層)を構成.MUC5ACがBA悪化と関連(DPBでも増加する).
\item ダニとハウスダストのIgEは95\%以上かぶる.舌下免疫療法の適応は5歳以上.
\item AERDは1:2で女性に多い,20-40歳.IgE介在しない.
\item ピークフロー(PEF):最大呼気流速を測定.コントロールの目標は日内変動<予測値の20\%.
\item 気道可逆性検査:FEV1測定後,SABAを吸入し,15-30分後に再度FEV1を測定.改善率>12\% and 改善量≥200mL で気道可逆性あり.
\item BAの臨床的寛解の基準:3コンポーネントなら①ACT≥23,②増悪なし,③OCS連用投与なし.4コンポーネントなら①②③+④「気管支拡張薬使用後,\%FEV1≥80\% で,かつ\%FEV1やPEFの日内変動<10\%」.
\item Bio適応疾患:慢性蕁麻疹はゾレア+デュピクセント(結節性痒疹はデュピクセントonly),慢性副鼻腔炎はヌーカラ+デュピクセント,EGPAはヌーカラ+ファセンラ.
\item テオフィリンの血中濃度は5-15μg/mL,消化性潰瘍.
\item ファセンラはEGPAなら自己注射OK(BAは適応なし)!
\item depemokimab(抗IL-5抗体,6ヶ月に1回投与)は今後市場に出る.
\insfig{ba_bio.jpg}{0.7}{https://medical.nikkeibp.co.jp/leaf/mem/pub/report/202506/589032.html}

\end{itemize}

\subsubsection{COPD}

\begin{itemize}
\item COPDの疾患関連遺伝子としてSERPINA(AATDの原因遺伝子).
\item COPDに関連するサイトカインとしてIL-17A,IL-1β,IL-6,TNF-α.
\item COPD患者ではMMP↑,プロテアーゼ活性↑,好中球エラスターゼ↑.Tc1型CD8+T↑,Th1型CD4+↑,Th17型CD4+↑.
\item Brinkman Index = pack-years * 20.
\item COPDの病期分類(GOLD分類):FEV1\%<70\%の条件のもとで,I期(\%FEV1≧80\%),II期(50-80\%),III期(30-50\%),IV期(<30\%).
\item CATスコアは軽症<10点,中等症10-20点,重症21-30点,超重症>30点.

\insfig{cat.jpg}{0.5}{https://www.gold-jac.jp/}

\item COPDにおけるNPPV導入基準:①動脈血pH≦7.35かつPaCO2≧45 Torr,②呼吸筋疲労,呼吸仕事量増大を伴う呼吸困難.
\item COPDでNPPV忍容性がない場合に在宅HFNCが可能.
\item COPDに対する新規治療薬:エンシフェントリン(PDE3/4阻害,吸入),ロフルミラスト(PDE4阻害,経口).
\item COPDに対するデュピクセントの適応は,「ICS+LABA+LAMAで増悪があり,Eosino≧300/μL」.
\item COPD患者へのMLs (CAM, AZM)は増悪を抑制する,ただし保険適応外.
\item AATは常染色体劣性遺伝,AATインヒビター(リンスバッド®)補充療法が保険適応.
\item ACOはGOLDでは病名として定義されておらず,COPDのフェノタイプとしてCOPD-A(COPDと喘息合併)を定義している.
\item ※肺嚢胞と肺気腫は別.肺嚢胞は喫煙と関連しない.SjDに合併.スキューバダイビングは制限.

\end{itemize}

\subsubsection{BE}

\begin{itemize}
\item BEの背景はRA, SjD, GERD, PCD, CFなど.
\item PCDは常染色体劣性遺伝,スクリーニングはサッカリンテスト(人工甘味料を下鼻甲介に置いて,味を感じるまでの時間の遅延),鼻腔内NO異常低値.
\item BEのCT:気管支内腔が隣接PAの1.5倍以上.
\item BEの重症度評価(予後因子):FACED score,BSI.
\item BEの予後不良因子:Pseudomonas検出.\%FEV1<50\%,mMRC≧2.

\insfig{be_prognosis.jpg}{}{呼吸器ジャーナル Vol.72 no.2 2024}

\item non-CFのBEに対してDPP-1阻害薬(ブレンソカチブ)がFDA承認.
\item DPB:HLA-B54, 典型例では慢性副鼻腔炎を合併.好中球性炎症で,MUC5AC増加(ムチンのコア蛋白:BA増悪因子でもある).寒冷凝集素価上昇(>64倍),増悪は肺炎球菌やHibが多い.
\item CF:CFTR遺伝子変異,汗の塩化物イオン濃度上昇.合併症:膵臓の外分泌機能不全,慢性副鼻腔炎,肝硬変,先天性両側精管欠損症.トブラマイシン吸入は保険適応.
\end{itemize}


\subsubsection{BO}

\begin{itemize}
\item 環境因子:NO2, SO2, アスベスト吸入,薬剤性:アマメシバ,D-ペニシラミン,感染症:RS, アデノ,マイコプラズマなどもある.続発性:SjS,悪性リンパ腫,移植.
\item 造血幹細胞移植後の慢性GVHDとして出現(中央値14ヶ月,5-10\%,末梢血移植の方が多い),急性GVHDあるとリスク上昇.→なので移植後は定期的にPFT必要.
\item 臓器移植後にBO起こる,特に肺移植後にもよく起こる(50-60\%).BALでNeu主体の細胞数増多,ステロイド効果なし.根治治療は肺移植のみ(矛盾してるけど).
\item BOのCTでは肺野透過性は亢進.Air trappingの検出のため呼気CT撮る.中枢気道の気管支拡張は進行期まで起こらない.
\item 病理はconstructive bronchiolitis,病変間に正常気道が介在する.
\end{itemize}


\subsubsection{IP}

\begin{itemize}
\item 指定難病としての「IIPs」に含まれる疾患は,IPF, NSIP, RB-ILD, DIP, COP, AIP, LIP, PPFE, UCIP.
\item IIPsの指定難病の医療費助成対象は重症度III度以上.
\item 6MWTでmin SpO2<90\% まで下がれば無条件でIII度 = 医療費助成.
\insfig{ipf_severity.jpg}{}{https://pro.boehringer-ingelheim.com/jp/product/ofev/update-of-diagnostic-criteria-for-iips}
\item IPの最新の分類(ERS/ATS 2025)での変更点は,

① IIPsとsecondary IPをまとめて分類,

② まず間質性パターン(interstitial)と肺胞充填性パターン(alveolar filling)に分類し,間質性パターンをさらに線維化と非線維化に大別,

③ 間質性・線維化パターンの病型としてUIP,NSIPと細気管支中心性間質性肺炎(bronchiolocentric interstitial pneumonia, BIP)の3つに分類,

④ BIPは名前の通り細気管支を主体とした病変で,HP,CTD-ILD,DILDなどを包括的に含む,

⑤ AIP→特発性DADに名称変更,

⑥ DIP→肺胞マクロファージ肺炎(alveolar macrophage pneumonia, AMP)に変更.肺胞腔内のマクロファージの充満,喫煙と関連,

⑦ 肺胞充填性パターンの病型としてOP,RB-ILD,AMP,その他(AEP,CEP,PAP,リポイド肺炎),

⑧ MDD診断で確信度≧90\%(condifent),51-89\%(provinsional),≦50\% (unclassifiable)に分類.

\end{itemize}

\subsubsection{肺癌}

\begin{itemize}
\item OncomineだけOK:HER2(エンハーツ,ゾンゲルチニブ)
\item Oncomineだけ不可(Amoy/肺がんCPはOK):ルマケラス
\item AmoyだけOK:新規ROS1(タレトレクチニブ,レポトレクチニブ),新規MET(グマロンチニブ)\footnote{コンパニオン診断は最新情報を確認 https://hokuto.app/post/ZXTQ93WlRz5xYEUeiWoG}
\item 肺がんCPだけ不可(Amoy/OncomineはOK):Ami/Laz,ローブレナ,ロズリートレク,タブレクタ
\item NTRKのロズリートレク,ラロトレクチニブはF1のみコンパニオン診断OK.
\item G-CSFの一次予防はFN発症率>20\% のレジメンに適応.
\end{itemize}


\subsubsection{IPの鑑別疾患}

\begin{itemize}
\item リポイド肺炎:脂肪貪食マクロファージ,外因性肺炎,ガソリンや灯油,好物など.4割無症状.TBBで判定.対症療法.
\end{itemize}


\begin{itemize}
\item 中皮腫はヒアルロニダーゼ消化試験陽性,微絨毛がみられる,CEA陰性


\item 好酸球性肺炎のメディエーターとしてeotaxin.
\item IPFのBAL所見,本邦における重症度分類,難病指定のための要件,GAPモデル
\item PFDの副作用は光線過敏,NTDの副作用は下痢,消せ戦塞栓症
\item AE-IPFの診断基準
\item IPAFの診断基準.臨床ドメインと血清学ドメインと形態学ドメイン Raynaudなどは臨床ドメインか?
\item COPは性差なし.non-smoker多い.CD4/8低下.
\item PPFEはCYなどの悪性腫瘍,GVHDとして発症
\item ARDSのベルリン定義:侵襲または呼吸器症状出現から1習慣以内,両側性の陰影,PEEP 5cm以上でP/F<300,200と100をカットオフで中等症と重症.小児では片側でもARDSで診断.
\item ARDSは予測体重を用いて6-8mL/kgの換気.筋弛緩は<48hrで終了.好中球エラスターゼ阻害薬は一応保険収載.
\item RAではBOありえる.SjDではMALTomaありえる.SLEで肺胞出血の場合はPE.
\item IgG4RD診断基準.
\item サルコイドーシスでCD4/8↑.
\item LCHでCD1a陽性細胞,S-100蛋白陽性.
\item GPAでステロイド+CY.
\item アミロイドーシスの胸膜病変で胸水貯留
\item maltomaの合併:H.pylori, SjD, 橋本病
\item 生着症候群は自家移植.30-100日だとCMV, トキソプラズマ,ARDS,リンパ増殖性疾患,PH.100日すぎるとBOやPPFE
\item 超合金肺の原因はコバルト,タングステン
\item 石綿肺→10-20年経って生じる.良性石綿胸水()→びまん性胸膜肥厚.いずれも高濃度暴露.
\item 低濃度暴露で生じるのは胸膜プラーク→MPM
\end{itemize}

\subsubsection{石綿とMPM}
\begin{itemize}
\item 石綿の高濃度曝露→良性石綿胸水,びまん性胸膜肥厚.低濃度曝露→胸膜プラーク,MPM.
\item 良性石綿胸水の定義:数ヶ月で自然軽快,胸水確認後3年は悪性腫瘍がない(両側に生じてもOK).
\end{itemize}


\subsubsection{じん肺の管理}

\begin{itemize}
\item じん肺の管理は管理1-4に分類され,管理2以上のものは健康管理手帳,公費で健診受けられる.
\item じん肺の診断時にDrが行う検査は①胸部Xp ②PFT → 労働局へ提出.
\item \textbf{じん肺一般の}労災基準:管理4全員,管理2/3でTb/Tb胸膜炎,続発性気胸/気管支炎,気管支拡張症, 肺癌.
\item \textbf{石綿の}労災基準:じん肺の労災基準を満たす場合(じん肺ではなく特に石綿肺と呼ぶ),中皮腫,良性石綿胸水,びまん性胸膜肥厚,肺癌(石綿小体の数が重要なことがありBALはできるだけ行う)
\item 石綿関連疾患で職業曝露以外だと労災は受けられないが,石綿健康被害救済制度の対象となる.MPM,石綿肺癌,良性石綿胸水,びまん性胸膜肥厚いずれも適応になる.
\item 立位では肺尖部で肺胞気圧>肺AV圧.TXA2は血管収縮作用.基本的に徐脈の治療→β刺激薬→気管支拡張作用
\item 有効血管床の50-60\%が障害されるとPA圧上昇
\item 肺性心のECG所見:V1-V3のR波増高,V4-V6深いS波,II, III, aVFのP波高.
\item PE: S1Q3T3
\item 高地肺水腫→低酸素性の肺血管攣縮.神経原性肺水腫→交感神経活性化による毛細血管圧↑
\item PAH: sGC↓エンドセリン↑eNOS(NOが血管拡張作用)↓PDE-5↑PGI2↓
\item 骨髄増殖性疾患によるPHは5群.薬剤性PHの原因:食欲抑制薬(網のレックス,フェンフルラミン),ダサチニブ.PAHであればまずHIV,AUS(門脈圧亢進)
\item PHがあれば常にHOTOK???
\item PAH治療薬のまとめ
\item 3群PHには吸入トレプロスチニル,SSc-PHにも血管拡張薬,他は原疾患の治療から
\item Well's criteria 急性PEには初手はt-PAはあり,カテーテルはなし
\item cancer VTEに対するDOACsの適応について
\item CTEPH 適切な抗凝固治療を3-6m継続しても慢性的にPHが継続している場合にCTEPHと診断.70歳台発症ピーク.DVTや血栓素因などで肺塞栓を反復的に繰り返す.永続的な抗凝固治療が必要.外科的治療は内膜摘除 or バルーン拡張.cTEPHにはワルファリンとリオシグアトは使える!!(最新は?)
\item 遺伝性出血性毛細血管拡張症:粘膜や皮膚の毛細血管拡張病変が特徴,肺動静脈奇形がある場合はダイビングNG.門脈圧亢進症や肝AV瘻によるPHもありえる
\item 肺葉内肺分画症(正常肺葉内に存在,PVに還流,気道感染反復),肺葉外肺分画症(固有の臓側胸膜を持つ,合併奇形あり,無症状),
\item 組織学的効果判定のgrade
\item 免疫染色態度,CD7+CK20-TTF1+,CK7-CK20+だと大腸がんを疑う.p40+, CL5/6+だと扁平上皮癌疑い
\item 炎症性偽腫瘍でALK陽性
\item CT followの頻度について
\item OSASはエプワース眠気尺度,ベルリン質問し,STOP質問しなど
\item PSGはAHI>20,簡易モニターでAHI>40でもOK
\item CPAPでも日中の眠気が続く場合はモダフィニル使用できる
\item PSGで何をモニターしているか:脳は,心電図,筋電図
\item CPAPは4時間以上を目標
\item CSAだと酸素,CPAP,CPAP装着下でAHI>15ならASV(adaptive servo ventilation)を考慮する
\item LAM細胞のクラスターは乳び胸水や乳糜腹水にみられる.LAM細胞はD2-40,α-SMA,HMB45,ER,PRなどの免疫染色が有用.血清VEGF-D
\item BHDS 顔面,頸部,上半身に繊維毛包腫.肺嚢胞は下肺野縦隔側,腎腫瘍はchromophobe腫瘍またはoncocytoma.folliculin遺伝子の異常でAD.
\item 先天性PAPの原因としてSP surfactant protein-B, SP-C, ABCA3遺伝子異常.PAPはBAL液中に泡沫状マクロファージ,末梢気腔内にPAS染色,SP-A染色陽性の再k粒状物質がみられる.PAPはGM-CSF吸入療法(保険適用)
\end{itemize}




\subsubsection{稀な疾患群}
\begin{itemize}

\item Platypnea-orthodeoxia症候群:心内シャント(卵円孔開存),V-Q mismatch(肝肺症候群,IP).

\item 黄色爪症候群(Yellow nail syndrome, YNS):黄色爪,四肢(下腿)リンパ浮腫.呼吸器症状の合併が多い(BE, 慢性気管支炎).ブシラミン,金,D-ペニシラミンと関連.
\item YNSの胸水:リンパ球優位,TG上昇.
\item YNSの治療は対症療法のみ:MLs, vitamin B3/E, 亜鉛.

\item MCDはVEGF↑.AAアミロイドーシス合併,HHV-8
\item 肝肺症候群:肺内シャントの存在はコントラスト心エコーが有用,3心拍以内にLAに気泡が到達した場合は心内シャントの存在が示唆される.
\item 肺胞出血の薬剤:抗癌剤(MTX,マイトマイシンC,CyA),抗てんかん薬(フェニトイン,カルバマゼピン),抗不整脈薬(アミオダロン,キニジン)

\item GPA palisading granuloma 中心部の壊死に対して柵状に組織球と巨細胞が取り囲む所見
\item 肺胞微石症:SLC34A2(リンの運搬蛋白),AR,リン酸カルシウムを主成分とする微小結石が蓄積

\item 喫煙は自然気胸の最大のリスク

\item Tb胸膜炎は一次結核に多い
\item 胸管損傷による乳び胸は胸管の破綻が第5胸椎より上で生じると左胸水,それより下だと右胸水になる.オクトレオチド.排液量が1000mL超える場合は胸管塞栓術やっていい.乳び胸の胸水はリンパ球優位.乳糜状況水(chyliform effusion)は胸腔内での炎症の遷延で生じる,CMは増加しない

\item MPMは胸膜80-85\%,腹膜10-15\%,MPMの70\%で石綿暴露あり
\item MPMのセルブロックの免疫染色.carretinin, WT-1, D2-40, CD5/6.MPMとと反応中皮過形成の鑑別は間質,脂肪組織への浸潤の有無を評価
\item desminによる免疫染色で,非腫瘍性中皮で80\%が陽性,上皮型中皮腫で10\%が陽性.
\item MPMの手術:壁側胸膜と肺表面の臓側胸膜のみを切除

\item 肺吸虫症はイノシシ,ウエステルマンが最多,腹腔→胸腔→気胸・肺野病変
\item 膿胸関連リンパ腫(PAL):慢性膿胸,結核性胸膜炎への人工気胸術後>20年,胸腔内に発生するDLBCL.
\item 原発性滲出性リンパ腫(PEL):HIV関連,HHV-8.
\item MPMガイドライン読む

\end{itemize}


\subsubsection{胸腺腫,胸腺癌}
\begin{itemize}
\item 胸腺腫+低γグロブリン→Good症候群で予後不良.
\item 胸腺腫の正岡分類:IIB期で肉眼的浸潤,IVA期で胸膜・心膜播種,IVB期でリンパ行性,血行性転移.
\insdualfig{thymoma_masaoka.jpg}{1}{thymoma_who.jpg}{1}{朝倉内科学(第12版)}\item 切除可能な胸腺腫・胸腺癌に針生検は禁忌.
\item 胸腺腫はIII期まで1st choiceは外科的切除.不完全切除例はPORTを行う(ただしIII期胸腺腫は完全切除でもPORTやってもいい).


\item 胸腺癌もIII期まで1st choiceは外科的切除.II-III期は完全切除でもPORT必須.
\item 切除不能胸腺腫は化学療法(ドキソルビシン,シスプラチン,CY±ビンクリスチン).ICIは治療関連死の可能性あり不可!
\item 切除不能胸腺癌は化学療法(CBDCA+PTX or CBDCA+AMR).2nd-lineはレンバチニブ.








\item セミノーマはhCG上昇(10-20\%),非セミノーマはAFP上昇(50-70\%).

\end{itemize}


\subsubsection{胸郭疾患}
\begin{itemize}
\item 横隔神経麻痺:第3-5頚椎から出る.診断は臥位Xp.
\item 横隔膜弛緩症:先天性は胎生期の形成不全.後天性は横隔膜の鈍的障害.左の挙上が多い.
\item 漏斗胸の手術適応の判断はHaller index:(胸腔横径)÷(胸骨後面と椎骨前面の距離)>3.25.


\item 大量血胸の開胸止血の適応:①胸腔ドレナージ開始時に1000mL,②開始1時間で1500mL,③開始2-4時間で>200mL/h,④持続的な輸血が必要.ただし鈍的外傷は開胸不要なことも多い.

\end{itemize}


\subsubsection{禁煙外来}

\begin{itemize}
\item 禁煙治療の条件:①禁煙意思あり,②ニコチン依存症スクリーニング>5点,③BI>200(35歳未満は制限なし),④文書による同意(+過去1年以内の禁煙治療歴なし)
\item バレニクリン(チャンピックス®)は漸増して12週使用.最初の1週間だけ喫煙OK!ドライバーには処方不可.
\item ニコチンパッチ(ニコチネル®)は漸減して8週使用.
\item 日本の禁煙治療の成功率は20\%.
\end{itemize}

\subsubsection{肺移植}

\begin{itemize}
\item レシピエントの年齢は両肺移植<55歳,片肺移植<60歳.
\item レシピエントがHIV抗体陽性でも移植可能.ひどい慢性気道感染があっても(一応)移植可能.
\item 除外条件は高度胸郭変形,T-Bil>2.5, Cre>1.5,肺外の活動性感染巣,悪性疾患,薬物依存,極端な肥満,現喫煙.
\item 肺移植後の移植片慢性機能不全(CLAD)も肺移植の適応となる.CLADの病型はBO(末梢気道の炎症,線維化と狭窄)またはRAS(Restrictive allograft syndrome:拘束性変化).


\end{itemize}
